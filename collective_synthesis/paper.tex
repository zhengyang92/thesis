%%
%% This is file `sample-sigplan.tex',
%% generated with the docstrip utility.
%%
%% The original source files were:
%%
%% samples.dtx  (with options: `sigplan')
%% 
%% IMPORTANT NOTICE:
%% 
%% For the copyright see the source file.
%% 
%% Any modified versions of this file must be renamed
%% with new filenames distinct from sample-sigplan.tex.
%% 
%% For distribution of the original source see the terms
%% for copying and modification in the file samples.dtx.
%% 
%% This generated file may be distributed as long as the
%% original source files, as listed above, are part of the
%% same distribution. (The sources need not necessarily be
%% in the same archive or directory.)
%%
%% The first command in your LaTeX source must be the \documentclass command.
\documentclass[sigplan]{acmart}
\settopmatter{printacmref=false, printfolios=false}
%%
%% \BibTeX command to typeset BibTeX logo in the docs
\AtBeginDocument{%
  \providecommand\BibTeX{{%
    \normalfont B\kern-0.5em{\scshape i\kern-0.25em b}\kern-0.8em\TeX}}}

%% Rights management information.  This information is sent to you
%% when you complete the rights form.  These commands have SAMPLE
%% values in them; it is your responsibility as an author to replace
%% the commands and values with those provided to you when you
%% complete the rights form.
\acmYear{2021}\copyrightyear{2021}
\setcopyright{acmlicensed}
\acmConference[PPoPP '21]{26th ACM SIGPLAN Symposium on Principles and Practice of Parallel Programming}{February 27--March 3, 2021}{Virtual Event, Republic of Korea}
\acmBooktitle{26th ACM SIGPLAN Symposium on Principles and Practice of Parallel Programming (PPoPP '21), February 27--March 3, 2021, Virtual Event, Republic of Korea}
\acmPrice{15.00}
\acmDOI{10.1145/3437801.3441620}
\acmISBN{978-1-4503-8294-6/21/02}

%%
%% Submission ID.
%% Use this when submitting an article to a sponsored event. You'll
%% receive a unique submission ID from the organizers
%% of the event, and this ID should be used as the parameter to this command.
%%\acmSubmissionID{123-A56-BU3}

%%
%% The majority of ACM publications use numbered citations and
%% references.  The command \citestyle{authoryear} switches to the
%% "author year" style.
%%
%% If you are preparing content for an event
%% sponsored by ACM SIGGRAPH, you must use the "author year" style of
%% citations and references.
%% Uncommenting
%% the next command will enable that style.
%%\citestyle{acmauthoryear}

%%
%% end of the preamble, start of the body of the document source.

\usepackage{tabularx}
\usepackage{booktabs}
\usepackage{subcaption}
\usepackage{xspace}
\usepackage{mathtools}
\usepackage{algorithm}
\usepackage[noend]{algpseudocode}

\input{macros.tex}

\begin{document}

%%
%% The "title" command has an optional parameter,
%% allowing the author to define a "short title" to be used in page headers.
\title{Synthesizing Optimal Collective Algorithms}

%%
%% The "author" command and its associated commands are used to define
%% the authors and their affiliations.
%% Of note is the shared affiliation of the first two authors, and the
%% "authornote" and "authornotemark" commands
%% used to denote shared contribution to the research.

\author{Zixian Cai}
\authornote{Both authors contributed equally to the paper. 
The work was done during internships at Microsoft Research.} %% \authornote is optional;

%\orcid{0000-0003-2262-2380}             %% \orcid is optional
\affiliation{
  \department{Research School of Computer Science}       %% \department is recommended
  \institution{Australian National University}           %% \institution is required
  \city{Canberra}
  \state{ACT}
  \country{Australia}                   %% \country is recommended
}
\email{zixian.cai@anu.edu.au}          %% \email is recommended

%% Author with single affiliation.
\author{Zhengyang Liu}
\authornotemark[1]
\affiliation{
  \department{School of Computing}              %% \department is recommended
  \institution{University of Utah}            %% \institution is required
  \city{Salt Lake City}
  \state{UT}
  \country{USA}                    %% \country is recommended
}
\email{liuz@cs.utah.edu}          %% \email is recommended



\author{Saeed Maleki}

\affiliation{
  \institution{Microsoft Research}           %% \institution is required
  \city{Redmond}
  \state{WA}
  \country{USA}                   %% \country is recommended
}
\email{saemal@microsoft.com}         %% \email is recommended

\author{Madanlal Musuvathi}

\affiliation{
  \institution{Microsoft Research}           %% \institution is required
  \city{Redmond}
  \state{WA}
  \country{USA}                   %% \country is recommended
}
\email{madanm@microsoft.com}         %% \email is recommended

\author{Todd Mytkowicz}
\affiliation{
  \institution{Microsoft Research}           %% \institution is required
  \city{Redmond}
  \state{WA}
  \country{USA}                   %% \country is recommended
}
\email{toddm@microsoft.com}         %% \email is recommended

\author{Jacob Nelson}
\affiliation{
  \institution{Microsoft Research}           %% \institution is required
  \city{Redmond}
  \state{WA}
  \country{USA}                   %% \country is recommended
}
\email{jacob.nelson@microsoft.com}         %% \email is recommended

\author{Olli Saarikivi}

\affiliation{
  \institution{Microsoft Research}           %% \institution is required
  \city{Redmond}
  \state{WA}
  \country{USA}                   %% \country is recommended
}

\email{olsaarik@microsoft.com}         %% \email is recommended
%

%%
%% By default, the full list of authors will be used in the page
%% headers. Often, this list is too long, and will overlap
%% other information printed in the page headers. This command allows
%% the author to define a more concise list
%% of authors' names for this purpose.
\renewcommand{\shortauthors}{Zixian Cai, Zhengyang Liu \etal}

%%
%% The abstract is a short summary of the work to be presented in the
%% article.
\begin{abstract}
Modern computer architectures are complex, with a wide range of
features that can be leveraged to optimize software performance.
However, efficiently optimizing software for these architectures
remains a challenging task. Traditional optimization techniques,
such as manual optimization by human experts or optimization by
compilers, may not fully exploit the unique features of novel
architectures, leading to suboptimal performance.

Program synthesis is a promising approach to address this challenge.
At its core, program synthesis searches for programs that meet a
specified set of requirements. Recent advances in SMT solvers and
increased computation power have made program synthesis a viable
choice for code generation. Program synthesis can generate code
specifically tailored to the target architecture, leveraging
domain-specific knowledge and advanced search techniques to create
highly optimized code.

The overall goal of this dissertation is to develop program
synthesizers that efficiently optimize software for emerging
architectures. The thesis statement of my dissertation is that program
synthesis can be used to generate highly optimized code for novel
architectures, outperforming traditional optimization techniques. To
achieve this goal, I develop program synthesizers that can
generate optimized code for specific architectures, leveraging
domain-specific knowledge and advanced search techniques. I
evaluate the effectiveness of these synthesizers by comparing the
generated code with manually optimized code and code generated by
traditional compilers. The results of this research demonstrate
the potential of program synthesis as a powerful tool for optimizing
software for emerging architectures.

I present Minotaur, a superoptimizer that uses program synthesis
to optimize LLVM IR code. Minotaur extracts program slices from LLVM IR
code, and uses an SMT solver to find optimized versions of these slices.
Minotaur is designed to work within the LLVM optimization pipeline, and
can be used to discover new optimizations that are missed by traditional
compilers.

We  present SCCL, a program synthesizer that optimizes collective
algorithms for parallel computation. SCCL uses domain-specific knowledge
about collective algorithms to generate highly optimized code for
specific architectures. SCCL is designed to work with MPI and OpenMP
applications, and can be used to improve the performance of parallel
applications on novel architectures.




\begin{comment}


The rapid evolution of computer hardware introduces novel
architectures that demand efficient utilization for optimal performance.
These innovative architectures enhance the capabilities of processors,
enabling them to perform more complex tasks and improve overall efficiency.
%
For example, GPUs achieve high performance by utilizing specialized hardware
features specifically designed to optimize parallel processing and
computational tasks, and the Vector Processing Units (VPUs) within CPUs
are designed to handle vector operations efficiently,
making them particularly well-suited to applications
 such as video encoding, compression,
and image processing applications.
%
These cutting-edge new architectures are gathering significant interest
from software developers and researchers.

Programs running on top of these new architectures are either
optimized manually by human experts, or optimized by optimizing compilers.
However, both approaches struggle to keep pace
with innovations in architecture, as
the features in those new architectures may not be fully supported
or understood by existing compilers and experts;
%
additionally, optimizing compilers and experts usually rely on heuristics
to determine the best optimization strategies.
%
These heuristics might not be tailored to the specific features
of novel architectures, leading to suboptimal performance.
%
Consequently, efficiently optimizing programs for new architectures remains
a challenging endeavor.

Program synthesis is a promising approach to address the challenge.
%
At its core, program synthesis searches for programs
that meet a specified set of requirements.
%
Nowadays, the advances in SMT solvers and increased computation power
have made program synthesis a more viable choice for code generation.
%
There are several reasons why program synthesis can help address the challenge:

\begin{itemize}
\item \textbf{Adaptability}: Unlike optimizing compilers or human experts,
program synthesis can generate code specifically tailored to the
target architecture.
%minotaur
For example, in superoptimizers,
once the semantics of an instruction is fed into the refinement checker,
it can automatically generate code that takes advantage of the instruction.

\item \textbf{Domain-specific optimizations}: Program synthesis can take
into account the unique characteristics of a specific domain,
such as collective algorithms in parallel computation,
to create highly optimized code.
%sccl
This domain-aware optimization further enhances the
performance of applications on new architectures.

\item \textbf{Continuous improvement}: As an architecture evolves and
new features are introduced, program synthesis techniques can be updated
to incorporate these advancements, ensuring that the generated code remains
optimized over time.
\end{itemize}

This proposal focuses on developing program synthesizers
that aim to efficiently optimize software for emerging architectures.
The thesis statement of my dissertation is

\end{comment}
\end{abstract}

%%
%% The code below is generated by the tool at http://dl.acm.org/ccs.cfm.
%% Please copy and paste the code instead of the example below.
%%

\begin{CCSXML}
<ccs2012>
<concept>
<concept_id>10010520.10010521.10010528.10010530</concept_id>
<concept_desc>Computer systems organization~Interconnection architectures</concept_desc>
<concept_significance>500</concept_significance>
</concept>
<concept>
<concept_id>10011007.10010940.10010971.10010972.10010973</concept_id>
<concept_desc>Software and its engineering~Cooperating communicating processes</concept_desc>
<concept_significance>500</concept_significance>
</concept>
</ccs2012>
\end{CCSXML}

\ccsdesc[500]{Computer systems organization~Interconnection architectures}
\ccsdesc[500]{Software and its engineering~Cooperating communicating processes}

%%
%% Keywords. The author(s) should pick words that accurately describe
%% the work being presented. Separate the keywords with commas.
\keywords{GPU, Synthesis, Collective Communication, Interconnection, Network}

%% A "teaser" image appears between the author and affiliation
%% information and the body of the document, and typically spans the
%% page.

%%
%% This command processes the author and affiliation and title
%% information and builds the first part of the formatted document.
\maketitle

\section{Introduction}
% Machine learning workloads imply novel topologies
Recent trends in machine learning towards training and serving large models together with the stagnation of Moore's-law-induced compute performance has led system designers to include novel high-bandwidth interconnect networks both within and across nodes in distributed clusters. For instance, a \dgxone server consists of two x86 processors and eight GPUs, interconnected by NVIDIA's NVLink network as shown in Figure~\ref{fig:dgx1-topo}. These networks' designs are motivated as much by the need to perform efficient \allreduce, a crucial primitive in machine learning, as well as by hardware considerations such as signal integrity, cooling and physical layout.
%\todo{only NVLink is shown in the figure, maybe reword the sentence or change the figure?
%Actually, I am not even sure if a socket contains THOSE four GPUs.}
A wide variety of similar accelerators with novel high-speed interconnects are used to train machine learning models today, including AMD's MI50 GPUs~\cite{mi50}, Graphcore's IPUs~\cite{graphcore} and Google's TPUs~\cite{tpu}.
%\todo{We mention distributed clusters but don't otherwise address them in the paper, I think}

% Hand-written communication primitives - what are the problems.
These novel topologies require novel communication kernels to maximize performance. Today these kernels are written and optimized manually. For instance, NVIDIA Collective Communication Library (NCCL) has two general algorithms for the supported operations such as \allreduce: a high-bandwidth ring algorithm and a low-latency tree algorithm. These implementations are manually written and they do not necessarily have the best performance for different topologies including \dgxone's. On one hand, repeating this manual effort for other communication primitives such as \alltoall or extending already implemented algorithms to a wide variety of hardware topologies is simply infeasible.
%\todo{maybe just say communication algorithms, as ring/tree algorithms are mentioned later}

On the other hand, optimizing these communication kernels for performance for each topology and buffer size is crucial. For instance, we found 30\% of the training time for the 8.3 billion parameter Megatron language model with model parallelism is spent inside \allreduce where each
buffer is of medium size (10-100MB). Also, for data parallelism, the communication buffers
could range from a few KBs (one layer) to a few GBs (the entire model).
We expect this wide range of sizes as large models are developed and trained on
larger distributed clusters.

%As machine learning models growing both in size and training complexity, the
%potential payoffs for such automation are significant in our current world. For
%example, when the 8.3 billion parameter Megatron language model is trained with
%8-way model parallelism on an NVIDIA DGX-1~\cite{megatronlm-arxiv}, 30\% of the
%training time is spent on communication.

%GPUs are used to accelerate a wide variety of tasks, from machine learning to
%simulations for engineering and physics. As the complexity of these tasks grows,
%there is a trend to pack an increasing number of GPUs inside a single node. This
%in turn places increasing importance on the mechanisms for GPU-to-GPU
%communication.

%In contrast with traditional inter-node networks, the interconnects for GPUs
%inside a node are often highly asymmetric. This is caused by a number of
%concerns, such as limitations on wire length and signal quality requirements for
%high-bandwidth links such as PCIe and NVLink as well as limitations on GPU
%placement due to cooling and physical layout. This results in in-node networks
%often not corresponding to any widely studied network topology (e.g., butterfly
%or hypercube). \todo{Give an example here?}

%Just like in traditional networks, the communication patterns have to be
%tailored to the topology for maximum performance. This has been done in isolated
%cases as, for example, NVIDIA Collective Communication Library (NCCL) implements
%algorithms optimized for their 8 GPU DGX-1 servers (see Figure~\ref{fig:dgx1-topo}). However, the wide variety of
%available hardware targets means that it is hard for a library to provide
%optimal algorithms for all configurations, which makes this an ideal target for
%automation.

\begin{figure}
\includegraphics[page=1,width=\columnwidth]{figures/topos.pdf}
\caption{NVLink topology of an NVIDIA DGX-1.}
\label{fig:dgx1-topo}
\end{figure}
%\todo{the grey box indicating a socket has a low contrast when printed. colours for two different rings should have higher contrast (blue and orange perhaps?)}

% explain our approach
In this chapter, we automatically synthesize high-performance communication kernels.
Given a topology, specified as a graph with bandwidth constraints on nodes and edges, and a communication primitive, specified as the pre- and post-condition on data location and computation on it, we generate~(Section~\ref{sec:synthesis}) a quantifier-free SMT formula that captures the set of all feasible algorithms that implement the primitive on the input topology.
Exploring this space to appropriately minimize the number of communication steps or decrease the granularity of communication at each step, is a computationally difficult problem. We exploit
an SMT solver to synthesize algorithms that explore this tradeoff along the Pareto frontier between latency-optimality and bandwidth-optimality.
For every solution from the SMT solver, we automatically generate and lower~(Section~\ref{sec:lowering}) high-performance implementations.


% We approach this problem as a synthesis problem. Collective communication
% primitives can be specified in terms of pre- and post-conditions on which
% processes data resides on and optionally where a user specified reduction
% function has been applied. We encode pre- and post-conditions as well as actions
% to move data between processes as quantifier-free SMT formulas, which we solve
% using Z3. We further impose constraints on bandwidth usage based on the topology
% of the specific machine we are targeting and limit the number of steps for the
% whole algorithm. This gives us a way to explore the entire space of possible
% algorithms for implementing a given primitive on the target hardware.

% how do we make it scalable
When using SMT, finding the right encoding can make all the difference for the
feasibility of an approach. This paper details the important design choices in
our encoding that help it scale to all of our hardware targets. We use the SMT
encoding for \broadcasting collectives, such as \broadcast, while for \reducing
collectives, such as \reduce, we employ a reduction back to the synthesis
problem for \broadcasting collectives.
%A key observation is that topologies have natural symmetries. Our encoding exploits this symmetry to efficiently explore the space of algorithms without sacrificing satisfiability.
This reduction generalizes a well known fact that some \reducing collectives may
be produced by inverting a \broadcasting one, e.g. \reduce by inverting \broadcast.

%This allows us to reuse
%the synthesis of certain primitives such as \allreduce, further improving our
%scalability.
%\todo{"time time-reversed mirror-image", huh? find easier to understand wording.}
%\todo{what do you mean by "modularize"? it's more like we reduce (no pun intended) the problem of reduction to broadcasting, which is the wording used later in bullet points}

% This informed an optimization in
% our encoding, that allows us to make reasoning about which GPUs data has been
% reduced implicit.

%\todo{This is quite vague right now.}

% The ability to control the number of steps an algorithm must execute in gives us
% a novel capability for trading off between latency and bandwidth optimality. By
% synthesizing a range of algorithms at different points in this space we can use
% the best algorithm for any given message size. While MPI implementations and
% NCCL both switch between algorithms based on message size, our synthesis
% approach allows us to do so in a much more fine grained manner. We show that
% this enables our algorithms to provide better performance than NCCL at all
% message sizes. \todo{Update this claim with truth.}

We implement our approach in a tool called \toollong{}~(\tool{}),
which probes the target hardware topology, synthesizes algorithms for
it using Z3~\cite{z3} and finally generates CUDA code that efficiently implements that algorithm.  These algorithms are
synchronous; at every step of the algorithm, one or more of the nodes
send and/or reduce data from others.
%\todo{"CUDA code"=>hardware dependent code? we have AMD GPUs. From saemal: AMD runs CUDA. OpenCL is another possibility but no one really uses it.}

% algorithmic novelty
Some of the algorithms we synthesize are novel, with no known counterparts in
the literature occupying the same latency-bandwidth tradeoff. For example, we
have produced a latency-optimal 2-step (4-step) algorithm for
the \allgather (\allreduce) primitive in the DGX-1 topology (Figure~\ref{fig:dgx1-topo}) and
a bandwidth-optimal 3-step (6-step) algorithm for the \allgather (\allreduce) primitive on the
same topology.  In addition to providing novel
algorithms, our approach informs us when a combination of bandwidth and number
of steps is \emph{not possible}. This makes our synthesis approach a tool for
probing the algorithmic properties that a given topology provides, which is
useful for co-design of hardware interconnects with communication libraries.
%\todo{"occupying the same latency-bandwidth tradeoff": exhibiting? and being different isn't necessarily good, we want more bandwidth with the same latency or lower latency with the same bandwidth.}
%\todo{tell reader that the "number of steps" correlate with latency?}
% results
Our evaluation~(Section~\ref{sec:evaluation}) shows us that this approach scales and beats NCCL in almost all cases.


To summarize, the contributions of this chapter are as follows:
\begin{itemize}
    \item A formalization of the synthesis problem for \broadcasting collectives.
    \item A general strategy for encoding the synthesis problem for
    collective communications algorithms into the quantifier-free linear integer
    arithmetic (QF\_LIA) sub-logic of the SMT-LIB logic.
    \item A reduction from the synthesis problem for \reducing collectives to that for \broadcasting collectives.
    % \item Explanations of some novel algorithms our synthesis has produced.
    \item A description of how \tool{} generates efficient code for the algorithms we synthesize on nodes with NVIDIA or AMD GPUs.
    \item An evaluation of \tool's generated algorithms on common server topologies for deep learning workloads and a comparison against NCCL.
\end{itemize}

%\todo{Paper structure paragraph?}

%%% Local Variables:
%%% mode: latex
%%% TeX-master: "paper"
%%% End:

\section{Overview}
This section provides an overview of synthesizing latency- and bandwidth-optimal algorithms, using \allgather for the 
\dgxone topology~(Figure~\ref{fig:dgx1-topo}) as the running example. 

\subsection{Collective Communication Primitives}
\label{sec:background-collectives}
Collective communication primitives allow nodes in a networked system to perform operations on shared data. As an example, if each node has some input data, the \allgather primitive transfers these data to all of the nodes.  One way to implement this is for each node to independently send its data to all other nodes. But, an algorithm in which the nodes collectively work together can be more efficient. The efficiency of such algorithms depends on the network topology. 

\begin{comment}
    For instance, given a set of nodes each having an array of data, the \allreduce primitive computes the sum (or some specified associative operation) of all these arrays and stores the result in each of these arrays. 

Computing the primitives such as \allreduce requires the nodes to communicate. This communication usually happens in smaller chunks of the input array. Given $N$ nodes, say we split the input array into $N$ chunks, where $c_{i,j}$ represents the $i$th chunk at node $j$. One way to compute \allreduce is as follows. First, we compute at node $i$, a partial sum $d_i = \sum_j c_{i,j}$. These reductions can be performed by arranging $N$ nodes in a spanning tree and communicating chunks $c_{i,j}$ along the tree. Note, there are $N$ parallel reductions each possibly using a different spanning tree. This arrangement of reduced data is called the \reducescatter primitive. Once we have the chunks $d_i$, we can perform an \allgather operation by ensuring each node has a copy of $d_i$. In essence, \allgather involves $N$ simultaneous broadcasts of $d_i$ from node $i$. Thus, we have implemented \allreduce by performing an \reducescatter followed by an \allgather.
\todo{I think the ``$N$ parallel reductions each possibly using a different spanning tree'' makes this confusing---is this describing an approach where each node reduces $1/N$ of the chunks?}
\todo{say "compute at node $i$, a partial sum" of chunk $i$.}
\todo{they're are a part of the sum of arrays. they are not really partial sums, in the sense that, e.g., only data from half of the nodes are added.}
\todo{"there are $N$ parallel reductions"=>these $N$ parallel reductions}

Implementing a collective communication primitive depends on the topology. For instance, when executing parallel reductions or broadcasts in the implementation above, the algorithm has to choose different spanning trees to utilize the bandwidth on all links in the topology. Similarly, the algorithm has to choose the ideal chunk size to use for communication. We will demonstrate these choices for the \dgxone topology described below. 
\todo{"Implementing" $\ldots$ efficiently}
\todo{"has to chose ": might have to choose}
\end{comment}

\subsection{Topology}
The network topology specifies how the nodes are connected with each other and the latency and bandwidth constraints on the links connecting them. Consider the \dgxone topology shown in Figure~\ref{fig:dgx1-topo}. It consists of $8$ GPUs (or nodes, in the above formalism) split into two groups $\{0,1,2,3\}$ and $\{4,5,6,7\}$. The nodes in each group are fully connected. In addition, there are four inter-group links as shown in the figure. These nodes are connected through 
NVLinks, with some nodes connected with two parallel NVLinks as shown in Figure~\ref{fig:dgx1-topo}.  

%There are two kinds of links shown in two different colors. The fast links, such as the one connecting nodes $0$ and $1$ have twice the bandwidth as the (relatively) slow links, such as the one connecting nodes $0$ and $2$. The figure represents the fast links with two arrows. Each link is bidirectional allowing concurrent sends and receives at full bandwidth.
%\todo{Remove the colors and the double arrow should make it clear}
%\todo{Also just mention that they are called nv2 and nv1 links}
%\todo{"split into two" groups, each associated with one CPU socket.}
%\todo{"All the nodes within a socket": all the GPUs in the same group}
%\todo{"inter-socket links" sound like QPI: maybe inter-group?}
%\todo{also check the usage in latency-optimal algorithm}
%\todo{I recommend talking about these in terms of groups rather than sockets to avoid the NVLink/PCIe/QPI confusion too -jn}
%\todo{I also recommend not talking about ``fast'' and ``slow'' links, since the diagram shows a fast link as 2 slow links (which is indeed what it is), and we then talk about using 6 logical rings; better to just say 6 links and 6 rings}

The \dgxone's design was heavily influenced by the need to do gradient reduction for machine learning workloads. Specifically, this topology forms two non-overlapping rings: one connecting nodes $\{0,1,4,5,6,7,2,3\}$ with two NVLinks per edge and another connecting $\{0,2,1,3,6,4,7,5\}$ with one NVLink per edge. These rings are bidirectional and thus form $6$ logical single-NVLink rings. The NCCL library implements \allgather by running $6$ simultaneous ring algorithms as we discuss below.  

%Given the fast ring has twice the bandwidth and each ring is bidirectional, this allows the implementation to utilize $6$ logical rings with the same bandwidth %to implement collection primitives 
%(2 from the slow ring and 4 from the fast ring).

\subsection{Cost Model}
%\todo{I think we need to define bandwidth and latency optimality here. My understanding is that a bandwidth-optimal algorithm is one that sends the minimal amount of data necessary to complete its operation, and a latency-optimal one uses the fewest number of communication rounds possible to complete its operation. Is that what we mean?}
  
%Before discussing how to build bandwidth- and latency-optimal algorithms for this topology, we will introduce a simple cost model. 
We will characterize the communication cost using the $(\alpha, \beta)$ model~\cite{hockney1994communication}. That is, sending a message of size $L$ along a link costs $\alpha + L\cdot\beta$ time. 
Here, $\alpha$ is the latency of communication and captures the {\em fixed} costs, such as the overhead of initiating a transfer or invoking a GPU kernel, 
and $\beta$ is the inverse bandwidth of the link and captures {\em per-byte} costs, such as copying data into system buffers. Li \etal{} extensively studies the transfer time of buffers with 
different sizes over numerous GPU interconnections\cite{alphabeta}. Their result show that with NVLinks, the transfer time stays almost constant up-to a large buffer size and only then it start to increase linearly. 
These results confirm that the $(\alpha,\beta)$ model is suitable for characterizing communication cost over NVLinks.

The cost of a collective algorithm for an input of size $L$ will be of the form $a\cdot\alpha + b \cdot L \cdot \beta$. We call $a$ the {\em latency cost} of the algorithm and $b$ the {\em bandwidth cost} of the algorithm. Given a class of algorithms that implement a collective on a given topology, an algorithm is {\em latency-optimal} ({\em bandwidth-optimal}) if no other algorithm in the class has a lower latency (bandwidth) cost. Usually, there is a tradeoff between the latency cost and the bandwidth cost when designing collective algorithms.  An algorithm with latency cost $a$ and bandwidth cost $b$ is said to be {\em Pareto-optimal} with respect to a class of algorithms if for every algorithm in the class with latency cost $a'$ and bandwidth cost $b'$, we have $a = a' \Rightarrow b' \geq b$ and $b = b' \Rightarrow  a' \geq a$.

%The $\alpha$ term represents the {\em fixed} cost of sending a message from the overhead of invoking a GPU kernel to the latency across the network. The $beta$ term represents from the cost of copying data across system buffers to the time taken to transmit the bytes at a given network bandwidth. For small message sizes, the cost is determined by the fixed cost $\alpha$, while for large message sizes, the cost is determined by the per-byte cost $\beta$. 
%\todo{we are going to have a single kernel launch, so many not the kernel launch cost?}
%\todo{give our estimation of $\alpha$ and $\beta$ cost for DGX1 and NVLinks.}
%\todo{the from $\ldots$ to $\ldots$ in a long sentence is confusing: consider "such as"}
%\todo{note: I replaced ``packet'' with ``message'', since alpha is more about the cost to initiate a transfer rather than to send a single 256-byte NVLink packet}

\subsection{Bandwidth-Optimal Algorithm for \dgxone}
\label{sec:motivation:bw-optimal}
As described above, the \dgxone topology has $6$ logical rings. \allgather for one ring can be implemented as follows. Each node simultaneously sends its data to the next node in the ring. In subsequent steps, each node stores the received data and sends it to the next node in the ring. In $7$ steps all nodes will have received data from all of the other $7$ GPUs. The $6$-ring algorithm is a generalization of this algorithm. Each node splits its data into $6$ chunks and executes the ring algorithm along each of the $6$ rings, with one chunk per ring. If $L$ is the size of the input data, each ring algorithm takes $7$ steps and communicates $\frac{L}{6}$ bytes. Thus, the cost of the $6$-ring algorithm is  
$$7\cdot \alpha + \frac{7}{6}\cdot L \cdot \beta$$

Each node has to receive at least $7 \cdot L$ amount of data, and it has an agglomerated incoming per-byte cost of $\beta/6$ (6 incoming NVLinks). Thus, any algorithm for \allgather has to take at least $\frac{7}{6}\cdot L \cdot \beta$ amount of time. Thus, this algorithm is bandwidth-optimal for the \dgxone topology. But can we do better with the latency cost? 

Using the techniques described in this paper, we have automatically synthesized an algorithm~(Section~\ref{fig:dgxone:syn}) with cost $$3\cdot \alpha + \frac{7}{6}\cdot L \cdot \beta$$ To the best of our knowledge, this algorithm was not previously known. Moreover, we prove that this algorithm is Pareto-optimal with respect to the class of algorithms we call $k$-synchronous algorithms~(Section~\ref{sec:ksync}).  

\subsection{Latency-Optimal Algorithm for \dgxone}
The next question is whether we can improve upon the latency cost of the synthesized algorithm. If each node communicates its data along a binary tree instead of a ring, it would take at least $3$ steps. Using the techniques described in this paper, we have automatically synthesized a better algorithm~(Section~\ref{fig:dgxone:syn}) with cost $$2\cdot \alpha + \frac{3}{2}\cdot L \cdot \beta$$
Since the \dgxone topology has a diameter of $2$, this algorithm is latency-optimal. To the best of our knowledge, a latency-optimal algorithm for the \dgxone was not previously known. This algorithm is Pareto-optimal with respect to the class of $k$-synchronous algorithms.  

\begin{comment}
While bandwidth optimal algorithms are suitable for large inputs $D \gg \alpha / \beta$, they are wasteful for small inputs. In such cases, we seek a latency-optimal algorithm. NCCL is 
capable of creating trees but on a \dgxone, the created trees are always a line graph and
therefore, they have the same latency as a ring.
\todofor{Saeed}{Does NCCL have a tree based allgather? If not, what's the tree like in allreduce? Is it binary? from saemal: no it does not. For allgather it doesn't even have
any other implementation other than a ring and for allreduce on DGX1, it only creates a ring with the tree. It is safe to say that on DGX-1 they use a ring always even though they have the capability of doing a tree}
\todo{"they are wasteful for small inputs"=>they might not give the smallest possible latency for small inputs}

The technique discussed in this paper synthesized a better algorithm that takes $2$ steps! To the best of our knowledge, this algorithm is not previously known. The algorithm works as follows. In the first step, each node sends its chunk to all its neighbors. For instance, node $0$ sends its chunk to its intra-socket neighbors $1$, $2$, and $3$ as well to its inter-socket neighbor $5$. Since each link is bidirectional communication from nodes do not conflict. At the end of step 1, each node has received chunks from all its intra-socket neighbors and one chunk from its inter-socket neighbor. In the second step, each node sends the chunk from its inter-socket neighbor to all its intra-socket neighbors. That is, node $0$ sends the chunk from $5$ to nodes $1$, $2$, $3$. 

This two-step algorithm does not subdivide its input data of size $D$. So, we will call this algorithm a \chunkstep{1}{2} algorithm. The time taken by this algorithm is 
$$2\cdot \alpha + 2\cdot D\cdot\beta$$
Since the diameter of this topology is $2$ and each node has to communicate with every other node, one cannot implement an \allgather in one step. Thus, for $D \ll \alpha / \beta$, this algorithm is optimal. We call such algorithms latency optimal. 

A natural question, again, is whether a faster latency optimal algorithms exist that can use the bandwidth more efficiently. For instance, the \chunkstep{1}{2} does not make use of fast bandwidth links as it only sends one chunk between a pair of nodes at each step. This paper shows that such an algorithm exist.
\todo{Not true, with rounds, we have a better algorithm.}

\subsection{Pareto Frontier}
\label{sec:pareto}

While the bandwidth-optimal algorithm is suitable for large input sizes and the latency-optimal algorithm is for small input sizes, many interesting inputs might be of sizes in between the two limits. How do design optimal algorithms in such cases? Let us assume that a \chunkstep{c}{s} algorithm implementing a collective on a given topology exists. Using a similar reasoning as above, this algorithm will take time 
$$ s\cdot \alpha + \frac{s\cdot D \cdot \beta}{c}$$
Intuitively, we need to reduce $s$ to improve the latency of the algorithm and increase $c$ to improve the bandwidth utilization of the algorithm, while the constraints of the topology and the implemented collective will restrict the feasible pairs $(c,s)$. For instance, as argued above, $s \geq 2$ and $\frac{s}{c} \geq \frac{7}{6}$ when implementing \allgather on the \dgxone topology. In other words, for a given $s$, $c \leq \left\lfloor\frac{6 \cdot s}{7} \right\rfloor$. 
\todo{maybe need to justify why all algorithms can be expressed as \chunkstep{c}{s} performance-wise. Suppose different ranks have different number of steps, which takes different durations. Even though you can find the smallest time slice and the corresponding greatest common divisor logical chunk size, the steps now are "logical" steps, and you don't always pay the $\alpha$ cost.}
\todo{our algorithms are BSP. With the concept of rounds, each step might take different amount
of time. Since 1 bw and 2 bw have 2 as their lcm, having equal chunks makes sense. Maybe we should have this in Section 2?}


By trading latency by increasing $s$, one can search for the largest $c$ for which a feasible \chunkstep{c}{s} exists. Each such algorithm maximizes the bandwidth utilization for a given latency. We can increase $s$ till we reach a bandwidth-optimal algorithm. Together, these form a {\em Pareto frontier} of feasible algorithms. Which of these algorithms to use will depend on the size of the input data $D$, the latency $\alpha$ of the network, and the bandwidth $\beta$ of the network. 

\subsection{Summary of the Paper}
In this paper, we propose a systematic method to synthesize algorithms in the Pareto-frontier spanning form the latency-optimal algorithm to the bandwidth-optimal algorithm for a given collective on an input topology. We characterize a class of algorithms that captures a broad set of known algorithms and prove Pareto-optimality of both known algorithms and synthesized new algorithms. We automatically generate an implementation of these algorithms that is competitive with manually hand-tuned communication kernels in use today. 

\end{comment}


%\subsection{Synthesizing Optimal Algorithms}
%\todofor{Olli}{Summarize our approach to synthesizing algorithms on the pareto frontier.}
%Given a topology, we automatically optimal algorithms in the pareto frontier.  Essentially a summary of the paper. 


\section{Synthesizing Optimizations}

For every cut extracted from an LLVM function, \minotaur{}'s goal is to
find a cheaper way to compute the value returned by that cut.


\subsection{Enumeration of Partial Symbolic Candidates}

\begin{table}[t]
  \centering
  \begin{tabular}{ r | l }
    \hline
    \textbf{Operation Type} & \textbf{Instructions} \\
    \hline \hline
    Unary integer ops & ctpop, ctlz, cttz, bitreverse, bswap \\
    Unary floating point ops & fneg, fabs, fceil, ffloor, frint, \dots \\
    Binary integer ops & add, sub, mul, udiv, sdiv, umax, umin, smax, smin\\
    Binary floating point ops & fadd, fsub, fmul, fdiv, frem, \dots \\
    Bitwise ops & and, or, xor, shl, lshr, ashr \\
    Data movement ops & extractelement, insertelement, shufflevector \\
    Comparisons & icmp, fcmp, select \\
    Conversions & zext, sext, trunc, fptrunc, fpext, fptosi, sitofp, fptoui, uitofp \\
    Intrinsic functions & 165 vector intrinsics \\
    \hline
  \end{tabular}
  \caption{List of operations generated by Minotaur}
  \label{tab:operations}
\end{table}


\minotaur{} creates \emph{partially symbolic} candidates where
instructions are represented concretely, but constants are symbolic.
%
Symbolic constants reduce the size of the search space without giving
up synthesis power.


\minotaur{} generates every candidate that fits into a configurable bound
on the number of new instructions to synthesize.
%
However, LLVM's \texttt{bitcast} instruction, which changes the type
of an SSA value without changing its representation, does not count
towards this limit.
%
This is because \minotaur{} takes a low-level, untyped view of values.
%
For example, it internally treats a 16-way vector of 8-bit values the
same as an 8-way vector of 16-bit values: both of these are simply
128-bit quantities.
%
This lack of type enforcement allows \minotaur{} to find interesting,
low-level optimizations such as those that use bitwise operations to
rapidly perform certain floating point operations.
%
(Example 5 in Section~\ref{sec:examples} shows one of these.)


\minotaur{}'s enumeration algorithm generates a wide variety of instructions,
including arithmetic, bitwise, data movement, comparisons, conversions,
and calls to intrinsic instructions.
%
Table~\ref{tab:operations} shows the list of operations that are
supported by \minotaur{}'s synthesis algorithm.



\subsection{Checking Refinement}

\minotaur{} uses Alive2 to eliminate every candidate that does not refine
the specification.
%
We modified Alive2 in such a way that symbolic constants get turned
into literal constants as a side effect of the refinement check which,
in our modified version, is posed to the solver as an exists-forall
call.
%
In other words, \minotaur{} asks the question: ``Does there exist a
valuation of the symbolic constants such that the synthesis candidate
refines the specification for all possible values of the inputs?''
%
When the refinement relation holds, the candidate is a feasible
replacement for the specification and, also, the model produced by the
SMT solver contains literal values for the symbolic constants, giving
a complete, sound optimization.


\begin {figure}[tbp]
  \centering
  \includegraphics[width=\linewidth]{figures/solve_literal.pdf}
  \caption{Example of synthesizing a rewrite that contains literal
    constants.  Purple nodes are instructions reused from the original
    cut; blue and orange nodes are synthesized instructions and
    literal constants.}
  \label{fig:synthesizing}
\end{figure}

Figure~\ref{fig:synthesizing} illustrates this process.
%
For the incoming cut shown in the top of the figure, \minotaur{}
enumerates possible rewrites that could be applied to the cut.
%
For a candidate that has literal constants, \minotaur{} will ask Alive2 to
emit a exists-forall query to find a model for the literal constants
such that the rewrite refines the cut forall the inputs.
%
If the query is satisfiable, \minotaur{} will apply the rewrite to the
cut.


\subsection{Identifying Profitable Rewrites}

Predicting throughput of code running on modern microprocessors is not
straightforward.
%
To help developers improve performance-critical code, the LLVM Machine
Code Analyzer (LLVM-MCA)~\cite{llvmmca} was created.
%
It is an interactive tool that emits a graphical depiction of pipeline
behavior, but its functionality can also be accessed programmatically,
and this is what \minotaur{} does.


We estimate the cost of a rewrite in two places.
%
First, after enumerating candidates but before performing refinement
checks, we sort them in order of increasing cost using LLVM's
TargetTransformInfo~\cite{tti}; a cost model that roughly captures
execution cost on the target, and is cheap to compute.
%
We do this to ensure that likely-beneficial rewrites are tested first,
since we often run \minotaur{} under a timeout for each optimization site.
%
Second, rewrites that have been confirmed to be refinements of the
original code are compiled to x86-64 object code and then analyzed
LLVM-MCA~\cite{llvmmca}, the machine code analyzer.
%
This cost model is more accurate but considerably slower than the
first one.
%
\minotaur{} only applies a rewrite if its estimated cost, using LLVM-MCA,
is lower than that of the original cut.


Although LLVM-MCA can estimate the cycle cost of LLVM functions, we instead use
the number of uOps (``micro-operations,'' a modern x86 processor's internal
instruction set) as the estimated cost.
%
This choice was driven by empirical data: after extensive
experimentation, we determined that, for our purposes, uOps are a
better performance predictor than cycles.



\subsection{Representing and Caching Rewrites}
\label{sec:rewrite}

\minotaur{} stores each potential rewrite as a pair: $(C, S)$
where $C$ is a cut, represented by a function in LLVM
Intermediate Representation (IR), and $S$ is a rewrite description---an
expression in \minotaur's own intermediate representation that describes a
different way to compute the return value of $C$.
%
Rewrite descriptions are directed acyclic graphs containing nodes that represent
operations, and edges representing data flow.
%
Although the elements found in \minotaur{} IR are similar to those found
in LLVM IR, we could not reuse LLVM IR to represent rewrites since
LLVM IR does not support incomplete code fragments, and also rewrites
must contain enough information to support connecting the new code in
the rewrite to code in the unoptimized function.


To support caching, rewrites must be serializable.
%
The cut $C$ can be serialized using existing LLVM functionality, and we
created a simple S-expression syntax for serializing the $S$ part.
%
Figure~\ref{fig:syntax} shows the syntax of the IR\@.
%
For example, if the returning value of $C$, a 32-bit instruction is
replaced by left shift by one bit position, the textual format for
the expression is \texttt{(shl (val i32 \%0), (const i32 1), i32)}.


Rewrites are cached in a Redis instance: this implementation choice
allows the cache to be persistent across multiple \minotaur{} runs and
also makes the cache network-accessible.
%
Synthesis can be done online---during compilation---but also
offline, in a mode where \minotaur{} extracts cuts into the Redis
cache but does not perform synthesis.
%
In this mode, compilation is only slowed down by a few percent.
%
\minotaur's offline mode is designed for batch processing.
%
In this mode, a separate program called \texttt{cache-infer} retrieves
cuts from the cache, runs synthesis on them, and stores any
optimizations that it discovers back into the cache.
%
Unlike the online mode, which runs synthesis tasks one after the
other, offline mode can run all synthesis jobs in parallel.



\begin{figure}[tbp]
  \begin{tabular}{r c l}
    \emph{Op} &::=& \emph{Inst} $\vert$ \emph{Constant} $\vert$ \emph{Value} \\
    \emph{Inst}  &::=& (\emph{UnaryOp} \emph{Op}, \emph{Type}) $\vert$ (\emph{BinaryOp} \emph{Op}, \emph{Op}, \emph{Type}) $\vert$ (\emph{Conversion} \emph{Op}, \emph{Type}) $\vert$\\
              && (insertelement \emph{Op}, \emph{Op}, \emph{Op}) $\vert$ (extractelement \emph{Op}, \emph{Op})  $\vert$\\
              && (\emph{Comparison} \emph{Op} \emph{Op}) $\vert$ (select \emph{Op}, \emph{Op}, \emph{Op}) $\vert$ (\emph{Intrinsic} \emph{Op}, \emph{Op}) $\vert$ \\
              && (shufflevector \emph{Op}, \emph{Op}, \emph{Constant}) \\
    \emph{Constant} &::=& (const \emph{Type} \texttt{number-literal}) \\
    \emph{Value} &::=& (val \emph{Type} \texttt{llvm-identifier}) \\

    \emph{Type} &::=& \emph{ScalarType} $\vert$ <elements $\times$ \emph{ScalarType}> \\
    \emph{ScalarType} &::=& i1 $\vert$ i8 $\vert$ i16 $\vert$ i32 $\vert$ i64 $\vert$ half $\vert$ float $\vert$ double $\vert$ fp128 \\
    \emph{BinaryOp} &::=& xor $\vert$ and $\vert$ or $\vert$ add $\vert$ sub $\vert$ mul $\vert$ udiv $\vert$ sdiv $\vert$ \\
                && ashr $\vert$ lshr $\vert$ shl $\vert$ umax $\vert$ umin $\vert$ smax $\vert$ smin\\
                && fadd $\vert$ fsub $\vert$ fmul $\vert$ fdiv $\vert$ copysign $\vert$ \\
                && fmaximum $\vert$ fminimum $\vert$ fmaxnum $\vert$ fminnum \\
    \emph{UnaryOp} &::=& ctpop $\vert$ ctlz $\vert$ cttz $\vert$ bswap $\vert$ bitreverse $\vert$\\
                      && fneg $\vert$ fabs $\vert$ fceil $\vert$ ffloor $\vert$ frint $\vert$ fround $\vert$ ftrunc $\vert$ fnearbyint $\vert$ froundeven \\
    \emph{Conversion} &::=& zext $\vert$ sext $\vert$ trunc $\vert$\\
                    && fptrunc $\vert$ fpext $\vert$ fptosi $\vert$ sitofp $\vert$ fptoui $\vert$ uitofp \\
    \emph{Comparison} &::=& eq $\vert$ ne $\vert$ ult $\vert$ ule $\vert$ slt $\vert$ sle $\vert$\\
                && oeq $\vert$ ogt $\vert$ oge $\vert$ olt $\vert$ ole $\vert$ one $\vert$ ord $\vert$ ueq $\vert$ ugt $\vert$ uge $\vert$ ult $\vert$ ule $\vert$ une $\vert$ uno \\
    \emph{Intrinsic} &::=& ssse3.phadd.d.128 $\vert$ avx2.pavg.b  $\vert$ \dots (165 intrinsics in total) \\
  \end{tabular}
  \caption{Syntax for Minotaur rewrites}
  \label{fig:syntax}
  % \Description[syntax]{Minotaur syntax}
\end{figure}



\subsection{Integration with LLVM}

\minotaur{} is loaded into LLVM as a shared library where it runs as an
optimization pass.
%
We arranged for it to run at the end of LLVM's auto-vectorization pipeline.
%
We call the Dead Code Elimination pass after Minotaur to
clean up the code.

\section{Code Generation}
\label{sec:lowering}
The prior section described a synthesis procedure for generating
Pareto-optimal algorithms. This section describes a tool called
\tool{} that implements this procedure and generates high-performance
collective implementations for both NVIDIA and AMD GPUs.


Every synthesized algorithm, at its core, is a sequence of commands
that describe \emph{what} data needs to be sent (i.e., which chunk),
\emph{where} it needs to be sent (i.e., a source and destination),
\emph{when} it needs to be sent (i.e., during which synchronous step),
and \emph{with} which chunk(s) it needs to be reduced. \tool{}
generates SPMD multi-process \CC{} code combined with CUDA kernels
that implement these commands.

Each GPU involved in the computation has its own code as part of a
top-level switch statement. Communication between GPUs is enabled
using CUDA IPC memory handles, which allows a GPU to access a remote
GPU's memory using shared pointers. Thus, communication between GPUs
simply involves writing data to appropriate buffers. However, there
are a few crucial choices that impact the communication performance.

%This mechanism chooses the fastest available hardware transport when
%there are multiple connections available. For example, on a \dgxone,
%NVLink will be used by default to communicate between GPUs; otherwise
%PCIe may be used. On the Gigabyte Z52 consisting of AMD GPUs, xGMI
%will be used for GPUs in the same ring; otherwise PCIe will be used.
%Section~\ref{sec:evaluation} describes the details.

%\subsection{Hardware Interconnects and the Software that uses them}
% Before we discuss how we generate code, we first enumerate the
% possible ways for GPUs to communicate with each other in modern
% hardware.

%\subsection{Interconnect usage details}

%Once the memory handles between connected GPUs are exchanged, there
%are multiple ways to enable data transfers.

\subsection{DMA engines and kernel copies:} Data may be moved either by
executing load or store instructions through a kernel, or by using a
specialized DMA engine via \texttt{cudaMemcpy}. A kernel copy allows
data movement and computation to be fused in a kernel while a DMA
engine has a higher initial $\alpha$ cost but may have higher
bandwidth, leading to a lower $\beta$ cost. On NVLink, DMA engine
bandwidth is about 10\% better than kernel copy bandwidth, due to
details of the wire-level protocol. Transfers are packetized, with
each packet including a header (containing address, error correction
data, etc.) and a variable-length payload. DMA engines are able to
emit maximum-sized packets, but kernel copy packets are limited to the
128-byte cache line size.

\subsection{Push and pull models:} Each DMA engine is located on a
particular GPU. Data movement between two GPUs can be executed by
either the receiver's DMA engine (a {\em pull} model) or by the
sender's DMA engine (a {\em push} model). Kernel copies have the same
two approaches. This may have performance implications due to the link
protocol: the push model only needs to send write request packets with
a payload, whereas a pull model first sends request packets and then
receives response packets with data. When communicating
bidirectionally, the request packets reduce the bandwidth available
for the response packets. Thus, even though the push model may require
extra memory, we have found it to be up to 10\% faster than the pull
model.

%Furthermore, when a \reducing collective receives chunks, it can
%reduce immediately after the receiver pulls the data whereas in the
%push model, the reduction can be computed lazily right before the
%reduced chunk is sent. Thus,

%Our experiments suggest that the push model is faster than pull.

\subsection{Single and multiple kernels:} One way to implement a
synthesized algorithm is by emitting several kernels, one per step,
which forces a global synchronization between steps and, as a
consequence, introduces large overheads. Alternatively, \tool{} fuses
all steps into one kernel and thus we implement the synchronizations
between GPUs as a fine-grained signal and wait mechanism with shared
flags. In our single kernel implementation, each chunk for each
connection has a dedicated flag; a chunk on a GPU is valid only when
the associated flag is set. There is a
\texttt{\_\_threadfence\_system()} between the data movement
operations and the operation to set the flag on the remote GPU signals
that the transfer is complete.

\subsection{Size and Number of Thread Blocks:} \tool{} dedicates a
given number of thread blocks to each link and for each step, it uses
the same number of thread blocks to communicate through that link. For
different input sizes, the number of thread blocks significantly
affects performance and in later sections we show how we empirically
search for the fastest configuration for various input sizes.

% \paragraph{CPU or GPU?} Data movement between the CPU's memory and
% the GPU's memory may be done either by the GPU or the CPU. Using the
% GPU is common: the same tradeoffs discussed above apply. The CPU's
% memory is mapped into the GPU's address space; when a DMA engine or
% load/store instruction accesses that region, the GPU's memory system
% generates read or write operations over the PCIe bus. It is also
% possible to map the GPU's memory into the CPU's address space and
% use load/store instructions on the CPU to access the GPU's memory;
% this is the goal of the GDRCopy library~\cite{gdrcopy}. Since these
% transfers do not need to pay the GPU kernel invocation cost, their
% latency can be lower than GPU-initiated transfers; however, because
% the GPU cannot be prefetched, the bandwidth is low compared to
% GPU-managed transfers.

% \paragraph{Framing overhead} One challenge in determining whether a
% communication scheme is using a link efficiently is determining what
% data transfer bandwidth to expect. Manufacturers often report link
% speeds in terms of raw bit rate, but once link framing overhead is
% taken into account, the rate at which user data is transferred will
% be lower.

% For instance, an NVLink 2.0 link as found on the V100 has a raw
% unidirectional bandwidth of just over 25 GB/s~\cite{nvlink2}.
% However, NVLink communication is structured in terms of 16-byte
% flits; each packet contains between one and three header flits
% (containing address, operation, acknowledgment, error correction,
% and other information), and up to 16 data flits~\cite{nvlink1}.
% Thus, the user-visible bandwidth will always be lower than 25 GB/s.
% The choice between using DMA engines or kernel copies affects this:
% the DMA engines are able to emit maximum-sized packets and thus we
% see a user-visible bandwidth of about 22 GB/s; for kernel copies the
% packets are limited to the 128-byte cache line size, leading to a
% user-visible bandwidth of about 20 GB/s.

% The other interconnects found in our configurations have similar
% properties. We omit the protocol details here. AMD's xGMI (Infinity
% Fabric) interconnect on the MI50 has a raw bit rate of 46
% GB/s~\cite{mi50}, and the peak link bandwidth we observe is
% approximately 33 GB/s. The PCIe 3.0 x16 links on the NVIDIA
% configurations have a raw bandwidth of 16 GB/s, but we measure a
% peak bandwidth of about 14 GB/s. The PCIe 4.0 links on the AMD
% configuration has a raw bandwidth of 32 GB/s, and we measure a peak
% bandwidth of about 27.5 GB/s.

% \todofor{Todd}{Explain general code generation approach, producing
% CUDA C++, general structure of generated code} \subsection{GPU to
% GPU Communication Methods}

% \todofor{Jacob}{Enumerate and explain all the different ways to
% communicate between GPUs. cudaMemcpy, kernel pull/push, gdrcopy etc.
% Performance considerations and tradeoffs between these.}

% \todofor{Saeed}{Explain which GPU to GPU communication methods we
% chose and how we implemented them.}

% \subsection{Lowering}

% \paragraph{Targeting PCIe}: \tool{} generates code driven by the
% CPU. Memory involved in a collective is pinned.  When possible, we
% exploit NUMA effects to register pinned memory on the socket that
% owns it.  We exploit gdrcopy\cite{gdrcopy} for low-latency transfers
% over PCIe of this pinned memory.  We extend gdrcopy to also enable
% reduction operations as the current codebase only supports send/recv
% and not addition as required in, for example, an Allreduce.  Because
% pinning memory is expensive, we cache pointers to buffers used in
% collectives.


% \subsection{Lessons in Low-level Communication Optimizations}

% \todofor{Zhengyang}{Enumerate all the system hacks that went in.}

%%% Local Variables: %% mode: latex %% TeX-master: "paper" %% End:

\section{Evaluation}
\label{sec:evaluation}

This section evaluates \minotaur{}.


\subsection{Correctness}

Every optimization discovered by \minotaur{} has been formally verified by
Alive2.
%
Even so, bugs might remain in the instruction semantics that we have
added to Alive2, in our cut extractor, in our rewrite mechanism, or in
Alive2 itself.
%
To defend against implementation errors, we have compiled numerous
open source applications using \minotaur, and then run those applications'
test suites, to ensure that they were not miscompiled.
%
Furthermore, we have compiled SPEC CPU 2017 using \minotaur{} and
used the SPEC drivers to ensure that all of its benchmarks behave
as expected.


\subsection{Effect of Depth Bounds in the Cut Extractor}
\label{sec:loops}

%1345 loops are integer only + vectorizable
%plot only shows 879 loops, these are the loops touched by minotaur.

% 2386 loops are integer / fp + vectorizable
% depthlimit 5: 26339 exprs (305 opt), 290 source changed, 667 min elapsed
% depthlimit 4; 24334 exprs

% \begin{figure*}[tbp]
%   \centering
%   \subfloat[Targeting Intel Cascade Lake; geomean=1.061x\label{plot:loops-intel}]{
%     \includegraphics[page=1,width=\linewidth]{figures/data.pdf}
%   }`
%   \hfill'
%   \subfloat[Targeting AMD Zen 3; geomean=1.021x\label{plot:loops-amd}]{
%     \includegraphics[page=2,width=\linewidth]{figures/data.pdf}
%   }
%   \caption{Speedups---estimated by LLVM-MCA---due to running \minotaur{}
%     on a loop micro-benchmark suite}
% \end{figure*}

It is important for \minotaur{} to extract cuts that are of an appropriate
size.
%
If they are too large, compile times suffer and also the SMT solver
can be overwhelmed, leading to timeouts; if cuts are too small, then
they form an insufficient basis for driving an optimization.
%
To determine a good value for $B$, the depth parameter to the cut
extraction procedure shown in Algorithm~\ref{alg:slicing}, we
performed an empirical study.
%
We started with FlexC's benchmark suite~\cite{woodruff2023rewriting},
a collection of 2,386 compilable, non-trivial C functions containing
loops from FFMPEG, FreeImage, DarkNet, xz, bzip2, and the LivermoreC
benchmark.
% \footnote{The loop data set was provided
% by Alexander Brauckmann and Michael O'Boyle at the University of
% Edinburgh, UK\@.  At present, no citable reference for this work
% exists.}
%
When compiled to LLVM IR, these functions contain a total of 123,062
instructions; thus, our cut extractor was invoked 123,062 times for
each depth bound.
%
We chose this code as the basis for our experiment because it is
derived from real applications while also being small enough to
keep compile times manageable (compared to, e.g., SPEC CPU 2017,
which is much larger).


\begin{figure}[tbp]
  \centering
  \subfloat[Unique cuts extracted\label{fig:loop-expression}]{
    \includegraphics[width=0.32\linewidth]{figures/spec/expression-count.pdf}
  }
  \hfill
  \subfloat[Unique opts. synthesized\label{fig:loop-optimization}]{
    \includegraphics[width=0.32\linewidth]{figures/spec/optimization-count.pdf}
  }
  \hfill
  \subfloat[Compilation time\label{fig:loop-buildtime}]{
    \includegraphics[width=0.32\linewidth]{figures/spec/compilation-time.pdf}
  }
  \caption{Evaluating the effect of varying $B$, the depth bound for
    cut extraction}
  \label{fig:loop}
\end{figure}


We then ran \minotaur{} on these functions with all depth bounds from
0--7, measuring the number of unique cuts that were extracted, the
number of optimizations found, and the compilation time.
%
We used a one-minute timeout for individual Z3 queries, and we also
gave \minotaur{} a total of up to five minutes to synthesize an optimized
version of each cut.
%
Figure~\ref{fig:loop} summarizes the results of this experiment.
%
The number of unique cuts that are extracted grows quickly with $B$,
but eventually begins to saturate simply because the functions being
compiled do not always have very long dependency chains.
%
The number of synthesized optimizations also grows quickly, but it
peaks when $B=6$ and then it decreases because the size of the cuts
causes many solver timeouts.
%
Finally, the total compile time increases smoothly with the depth
bound, eventually leveling off as most solver queries time out.


For the experiments in the rest of the evaluation section, we chose
$B=4$ because this gets pretty close to the maximum observed number of
optimizations without requiring exorbitant compile times.
%
It seems likely that there is room for improvement in this aspect of
\minotaur: perhaps the depth bound should be determined adaptively.
%
In this scenario, we would extract more and more components into the
cut, until either an optimization is found or else the solver begins
to time out.
%
We leave explorations of this nature for future work.


\subsection{Speedups for Benchmarks and Applications}

In this section, we show how \minotaur{} speeds up real-world benchmarks
and applications.

\paragraph{Experimental setup}
%
We used two machines for our evaluation: one with an Intel Xeon Gold
6210U processor running at 2.5\,GHz (this implements the Cascade Lake
microarchitecture~\cite{cascadelake}) and the other with an
AMD Ryzen 5950x processor
running at 3.4\,GHz (this implements the Zen~3 microarchitecture~\cite{zen3}).
The Intel machine supports the AVX-512 instruction set.
%
Both machines run Linux and were idle except for a single core running
our benchmarks.
%
To reduce the performance variation caused by frequency scaling, we
disabled turbo boost on the Intel machine and the core performance
boost on the AMD machine.
%
We invoked LLVM with the \texttt{-march=native} compilation flag to
ask it to take maximum advantage of processor features; we left other
compilation flags unchanged, except where noted.
%
All benchmarks are compiled at the \texttt{-O3} optimization level.
%
We set the timeout for Z3~\cite{z3} queries to one minute.
%
Finally, for each instruction that it tries to optimize, \minotaur{} gives
up if no solution is found within five minutes.


\paragraph{Benchmark selection}
%
We evaluate on SPEC CPU 2017%\footnote{\url{https://www.spec.org/cpu2017/}}
because it is a widely accepted standard
benchmark.
%
We only evaluate on the \emph{speed} subset of the SPEC suite, and we omit
648.exchange, 607.cactuBSSN, 621.wrf, 627.cam4, 628.pop2, 649.fotonik3d,
and 654.roms as they contain Fortran code.
%
We additionally use GMP, the GNU Multiple Precision Library, and libYUV,
which is used by Google Chrome/Chromium for manipulating images in the
YUV format.
%
We chose these libraries because they have been heavily tuned for
performance, they rely on loops, and they come with performance
benchmark suites that we could simply reuse.


\paragraph{Compile times}
%
Table~\ref{tab:compiletime} shows how long it takes \minotaur{} to process
our benchmarks, along with the number of potentially optimizable
values and the number of optimizations found.
%
In most cases, \minotaur{} found more optimizations when targeting the AMD
processor.
%
We believe this is because LLVM is more mature targeting
AVX2 than AVX512.
%
Solving queries with 256-bit vectors is also less likely to cause Z3
to timeout than are 512-bit vectors.
%
Minotaur is quite slow when it runs with a cold cache because it
performs a large number of solver queries.
%
However, with a warm cache, it is only 3\% slower than baseline \texttt{clang}.

\begin{table}[t]
  \centering
  \Small
  \begin{tabular}{| r | r r  r | r r | r r r | r r |}
    \hline
    \multirow{2}{*}{}& \multicolumn{5}{c|}{Intel Cascade Lake} & \multicolumn{5}{c|}{AMD Zen3} \\
    \cline{2-11}
    & \multicolumn{3}{c|}{Compilation time (min)} & \multicolumn{2}{c|}{Opt. found} & \multicolumn{3}{c|}{Compilation time (min)} & \multicolumn{2}{c|}{Opt. found}  \\
    \hline
    Benchmarks & cold cache & warm & clang & \# cut & \# opt. & cold cache & warm & clang & \# cut & \# opt. \\
    \hline\hline
    SPEC CPU 2017 & 2,337 & 3 & 3 & 109,177 & 2,683 & 2,580 & 3 & 3 & 114,612 & 2,820 \\
    \hline
    gmp-6.2.1 & 440 & < 1 & < 1 & 9,170 & 336 & 445 & < 1 & < 1 & 9,265 & 387\\
    \hline
    libYUV & 2,196 & < 1 & < 1 & 6,849 & 334  & 2,193 & < 1 & < 1 & 6,809 & 357 \\
    % \hline
    % OpenBLAS-0.3.26 & 554 & < 1 & < 1 & 8,683 &  & 670 & < 1 & < 1 & 9,182 & 156 \\
    \hline
  \end{tabular}
  \caption{Compile-time statistics}
  \label{tab:compiletime}
\end{table}

\paragraph{Optimizing GMP with \minotaur{}}

\begin{figure}[tbp]
  \centering
  \subfloat[Speedups on Intel Cascade Lake, geomean = 1.073x\label{plot:gmp-intel}]{
    \includegraphics[page=1,width=\linewidth]{figures/data.pdf}
  }
  \hfill
  \subfloat[Speedups on AMD Zen 3, geomean = 1.065x\label{plot:gmp-amd}]{
    \includegraphics[page=2,width=\linewidth]{figures/data.pdf}
  }
  \caption{GNU Multiple Precision Library (GMP) speedups}
  \label{fig:gmp}
\end{figure}


GMP provides a portable C-language implementation and then, for
several platforms, a faster assembly language implementation.
%
For this evaluation, we selected the C implementation, because \minotaur{}
works on LLVM IR and cannot process assembly code at all.
%
The benchmark suite that we used is
GMPbench.\footnote{\url{https://gmplib.org/gmpbench}}
%
Figure~\ref{fig:gmp} summarizes the results.
%
When \minotaur{} targets the Intel Cascade Lake processor, and when the
resulting executables are run on that same microarchitecture,
all the benchmarks sped up;
across all of the benchmarks, the mean speedup was 7.3\%.
%
The analogous experiment using the AMD Zen~3 microarchitecture
resulted in one benchmark slowing down, and the rest of benchmarks
speeding up, for an overall mean speedup of 6.5\%.


\paragraph{Optimizing libYUV with \minotaur{}}


\begin{figure*}[tbp]
  \centering
  \subfloat[Speedups on Intel Cascade Lake, geomean = 1.022x\label{plot:libyuv-intel}]{
    \includegraphics[page=3,width=\linewidth]{figures/data.pdf}
  }
  \hfill
  \subfloat[Speedups on AMD Zen 3, geomean = 1.029x\label{plot:libyuv-amd}]{
    \includegraphics[page=4,width=\linewidth]{figures/data.pdf}
  }
  \caption{LibYUV speedups}
  \label{fig:yuv}
\end{figure*}


This library has an extensive test suite, part of which is explicitly
intended for performance testing; we used this part as a benchmark.
%
Each of them scales, rotates, or converts a 1280\,x\,728 pixel
image 1,000 times.
%
Figure~\ref{fig:yuv} shows the results of this experiment.
%
When \minotaur{} targets an Intel processor, $148$ programs slowed down, $72$
did not change performance, and $2,312$ sped up, for an overall speedup of
2.2\%.
%
Targeting an AMD processor, $188$ programs slowed down, $85$ did not
change performance, and $2,259$ sped up, for an overall speedup of 2.9\%.
%
\minotaur{} can make code slower because it looks at optimizations in
isolation; it does not attempt to model interactions between
optimizations.


libYUV is portable code, but it has already been heavily tuned for
performance; most commits to its repository over the last several
years have been performance-related.
%
Our hypothesis is that this manual tuning has already eaten up most of
the performance gains that we would have hoped to gain from \minotaur{}.
%
For some time now, Google's released versions of Chrome have been
compiled using LLVM; the Chrome engineers have had ample time to
ensure that this compiler achieves decent code generation for
performance-critical libraries.

% \begin{figure*}[tbp]
%   \centering
%   \subfloat[Normalized speedup; geomean = 1.013x\label{fig:spec-intel-speed-ups}]{
%     \includegraphics[width=0.45\linewidth]{figures/spec/spec-intel.pdf}
%   }
%   \hfill
%   \subfloat[Compilation time in seconds\label{fig:spec-intel-compilation-time}]{
%     \includegraphics[width=0.45\linewidth]{figures/spec/spec-intel-compiletime.pdf}
%   }
%   \caption*{Targeting Intel Cascade Lake\label{fig:spec-intel}}
%   \hfill
%   \subfloat[Normalized speedup; geomean = 1.012x\label{fig:spec-amd-speed-ups}]{
%     \includegraphics[width=0.45\linewidth]{figures/spec/spec-amd.pdf}
%   }
%   \hfill
%   \subfloat[Compilation time in seconds\label{fig:spec-amd-compilation-time}]{
%     \includegraphics[width=0.45\linewidth]{figures/spec/spec-amd-compiletime.pdf}
%   }
%   \caption*{Targeting AMD Zen 3\label{fig:spec-amd}}
%   \caption{SPEC CPU2017 benchmark performance and compilation time}
%   \label{fig:spec}
% \end{figure*}

\begin{figure}[tbp]
  \centering
  \subfloat[Speedups on Cascade Lake; geomean = 1.015x\label{fig:spec-intel-speed-ups}]{
    \includegraphics[width=0.48\linewidth]{figures/spec/spec-intel.pdf}
  }
  \hfill
  \subfloat[Speedups on Zen 3; geomean = 1.012x\label{fig:spec-amd-speed-ups}]{
    \includegraphics[width=0.48\linewidth]{figures/spec/spec-amd.pdf}
  }
  \caption{SPEC CPU2017 benchmark performance}
  \label{fig:spec}
\end{figure}

\paragraph{Optimizing SPEC CPU2017 with \minotaur{}}

Figure~\ref{fig:spec} shows the effect of optimizing the benchmarks
from SPEC CPU2017 using \minotaur.
%
When optimizing for, and running on, the Intel processor, we observed
a mean speedup of 1.5\%.
%
When optimizing for, and running on, the AMD processor, we observed a
mean speedup of 1.2\%.
%
It is notoriously difficult to speed up the SPEC CPU benchmarks
because compiler engineers have already put considerable effort into
achieving good code generation for them.



\subsection{Optimizations Discovered by \minotaur}
\label{sec:examples}

The purpose of this section is to examine \minotaur's strengths by
presenting some optimizations that it found while compiling benchmark
programs.
%
None of these optimizations can be performed by the version of LLVM
that \minotaur{} is based on,\footnote{\minotaur{} uses LLVM~18.1.0 for all
results in this paper.}  at its \texttt{-O3} optimization level.
%
We present optimizations in an SSA format that is close to LLVM IR,
but we have edited it slightly for compactness and legibility.

\iffalse
One might be inclined to ask, while reading this section, ``Why is
LLVM incapable of performing this transformation?''
%
Alas, there is no single answer.
%
In some cases, performing the transformation would require the
optimizer to have a semantic model of a processor-specific intrinsic
function, but mostly these models do not exist.
%
In other cases, such as Example~5 below, generic reasoning about the
code would be very difficult, and a specific pattern matcher might not
be robust enough to be worth implementing.
%
Finally, our observation is that vector support in LLVM is somewhat
newer and less mature than support for other IR features, and the
optimizers have simply not had enough time to accumulate the requisite
optimizations.
\fi


\paragraph*{Example 1}

This code is from perlbench in SPEC:

{\small\begin{quote}\begin{verbatim}
%0 = zext <16 x i8> %x to <16 x i16>
%1 = zext <16 x i8> %y to <16 x i16>
%2 = call @llvm.x86.avx2.pavg.w(%0, %1)
%3 = trunc <16 x i16> %2 to <16 x i8>
ret <16 x i8> %3
  =>
%0 = call @llvm.x86.sse2.pavg.b(%x, %y)
ret <16 x i8> %0
\end{verbatim}
\end{quote}}

The unoptimized code zero-extends each 8-bit element of the two input
vectors to 16~bits, calls the AVX2 variant of \texttt{pavg} to perform
element-wise averaging of the extended vectors, and then truncates
elements of the resulting vector back to eight bits.
%
The optimized code simply calls an SSE2 version of the \texttt{pavg}
instruction that operates on 8-bit elements, reducing the uOp cost
of the operation from four to one.


\paragraph*{Example 2}
%https://godbolt.org/z/vjjr7MGzb

This code is from libYUV, ``... an open source project that includes
YUV scaling and conversion
functionality'':\footnote{\url{https://chromium.googlesource.com/libyuv/libyuv/}}

{\small\begin{quote}\begin{verbatim}
%0 = call @llvm.x86.avx2.pmadd.wd(%x, <0,1,0,1, ...>)
%1 = call @llvm.x86.avx2.pmadd.wd(%x, <1,0,1,0, ...>)
%2 = sub nsw <8 x i32> %1, %0
ret <8 x i32> %2
  =>
%0 = call @llvm.x86.avx2.pmadd.wd(%x,<1,-1,1,-1, ...>)
ret <8 x i32> %0
\end{verbatim}
\end{quote}}

The \texttt{pmadd.wd} (multiply and add packed integers) instruction multiplies
signed 16-bit integers element-wise from two input vectors, and then
computes its output by adding adjacent pairs of elements from the
resulting vector.
%
Thus, the input to this instruction is two 16-way vectors containing
16-bit elements, and its output is a single 8-way vector of 32-bit
elements.


In this example, the second argument to each \texttt{pmadd.wd}
instruction in the unoptimized code is a vector of alternating zeroes
and ones, which has the effect of selecting odd-indexed elements into
\texttt{\%0} and even-indexed elements into \texttt{\%1}.
%
Then, after the \texttt{sub} instruction, which simply performs
element-wise subtraction of \texttt{\%0} and \texttt{\%1}, the overall
effect of this code is to compute the difference between adjacent
pairs of elements of \texttt{\%x}.
%
\minotaur{} is able to perform this same computation using a single
\texttt{pmadd.wd} instruction which negates odd-numbered elements of
\texttt{\%x} before performing the addition.
%
The optimized code requires $5$ uOps to execute whereas the original
code requires $8$.


\paragraph*{Example 3}

This code is from libYUV:

%https://godbolt.org/z/7ooobqofK
{\small\begin{quote}\begin{verbatim}
%0 = shufflevector <32 x i8> %x, poison, <3, 7, 11, 15, 19, 23, 27, 31>
%1 = lshr %0, <6, 6, 6, 6, 6, 6, 6, 6>
%2 = zext 8 x i8> %1 to <8 x i32>
ret <8 x i32> %2
  =>
%0 = bitcast <32 x i8> %x to <8 x i32>
%1 = call @llvm.x86.avx2.psrli.d(<8 x i32> %0, 30)
ret <8 x i32> %1
\end{verbatim}
\end{quote}}

The \texttt{shufflevector} instruction in the unoptimized code selects
every fourth byte-sized element from the input \texttt{\%x}.
%
The resulting 8-way vector is right-shifted element-wise by six bit
positions, and that result is zero-extended to an 8-way vector of
32-bit elements.
%
\minotaur's optimized version (which executes in 4 uOps instead of 11)
first reinterprets the input vector's data as 32-bit elements; this
bitcast is relevant to LLVM's type system, but it is a nop at the CPU
level.
%
Then, the \texttt{prsli} instruction shifts each 32-bit element to the
right by 30 bit positions.
%
This right-shift-by-30 achieves the same effect as the unoptimized
code, where the \texttt{shufflevector} can be seen as a
right-shift-by-24, followed by an explicit right-shift-by-6.

\paragraph*{Example 4}

This code, from compiling perlbench from SPEC CPU 2017, illustrates
\minotaur's ability to reason about control flow:

% control flow divergence
%https://godbolt.org/z/e8jTsTMMz
{\small\begin{quote}\begin{verbatim}
entry:
  br i1 %c, label %body, label %if.end
body:
  br label %if.end
if.end:
  %p1 = phi [ %a, %body ], [ %b, %entry ]
  %p2 = phi [ %b, %body ], [ %a, %entry ]
  %r = call @llvm.x86.avx2.pavg.b(%p1, %p2)
  ret <32 x i8> %r
    =>
  %r = call @llvm.x86.avx2.pavg.b(%a, %b)
  ret <32 x i8> %r
\end{verbatim}
\end{quote}}

The intent of the code is to compute the element-wise average of input
vectors \texttt{\%a} and \texttt{\%b}, with a Boolean value
\texttt{\%c} determining the order in which the input vectors are
presented to the \texttt{pavg} instruction.
%
However, the order of arguments to this instruction does not matter, and
\minotaur's version executes in 4 uOps while the original code requires
10.
%
Note that \minotaur{} was not explicitly taught that \texttt{pavg} is
commutative; the necessary information was inferred naturally from the
formal specification.


\paragraph*{Example 5}

This is an optimization discovered
by \minotaur{} when it was used to compile GMP, the GNU Multiple Precision
Arithmetic Library, a widely-used library for arbitrary precision
integer computation:\footnote{\url{https://gmplib.org/}}

% before 19 after 13

{\small\begin{quote}\begin{verbatim}
%0 = lshr i64 %x, 1
%1 = and i64 %0, 0x5555555555555555
%2 = sub i64 %x, %1
%3 = lshr i64 %2, 2
%4 = and i64 %2, 0x3333333333333333
%5 = and i64 %3, 0x3333333333333333
%6 = add nuw nsw i64 %4, %3
%7 = lshr i64 %6, 4
%8 = add nuw nsw i64 %7, %6
%9 = and i64 %8, 0xf0f0f0f0f0f0f0f
ret i64 %9
  =>
%0 = bitcast i64 %x to <8 x i8>
%1 = call @llvm.ctpop(<8 x i8> %0)
%2 = bitcast <8 x i8> %1 to i64
ret i64 %2
\end{verbatim}
\end{quote}}

%
% \vspace{0.1in}
% %
% \caption{On the left, LLVM IR extracted from GMP; when compiled to
%   x86-64 code and run on an Intel Cascade Lake processor, its
%   predicted execution cost is 19 uOps. On the right, \minotaur's
%   optimized version of this code, which requires 13 uOps on the same
%   target.}
% \label{fig:ctpop}
% \end{figure*}
%
The original code performs a series of bit-level
manipulations on a 64-bit integer value, with the net result of
performing an 8-way vectorized 8-bit popcount operation.\footnote{The
popcount, or Hamming weight, of a bitvector is the number of ``1''
bits in it.}
%
The optimized code simply calls an intrinsic function to do the
popcount; it costs 13 uOps instead of the original code's 19.
%
Although robust recognition of open-coded idioms is not the focus
of our work, \minotaur{} does sometimes manage to achieve this.

Taking a strict view of types in the synthesis process could help
prune the search space, but it would also cause us to miss
optimizations that require a flexible view of types.
%
This example illustrates the latter case: the original code contains
no indication that a good optimization can be found using a vector of
type <8 x i8>, and therefore a strictly type-guided synthesis
procedure would miss this one.

\paragraph*{Example 6}

This code comes from 644.nab in SPEC CPU 2017:

% https://github.com/llvm/llvm-project/issues/85250
{\small\begin{quote}\begin{verbatim}
%0 = fcmp oge float %x, 0.000000e+00
%1 = fneg float %x
%2 = select i1 %0, float %0, float %2
%3 = fcmp oeq float %2, 0.000000e+00
ret i1 %3
  =>
%1 = fcmp oeq float %x, 0.000000e+00
ret i1 %oeq
\end{verbatim}
\end{quote}}

The original code computes the absolute value of a floating-point
number \texttt{\%x} and then checks if the result is zero.
\minotaur{} found that that the original code is equivalent to simply checking if
\texttt{\%x} is zero.


\paragraph*{Example 7}

This code is from the SPEC CPU 2017 benchmark 619.lbm:

% https://github.com/llvm/llvm-project/issues/85245
{\small\begin{quote}\begin{verbatim}
%0 = fsub float %x, %y
%1 = fcmp ogt float %0, 0.000000e+00
ret i1 %3
  =>
%0 = fcmp ogt float %x, %y
ret i1 %0
\end{verbatim}
\end{quote}}

The original code computes the difference between two floating-point
values, and then checks if the result is greater than zero. \minotaur{}
found that this code is equivalent to checking if the second value is
less than the first.


\paragraph*{Example 8}

This code comes from 619.lbm in SPEC CPU 2017:

% https://github.com/llvm/llvm-project/issues/85267

{\small\begin{quote}\begin{verbatim}
%0 = fmul float %x, 0x3FF0CCCCC0000000
%1 = fcmp olt float %t1, 0x3FE20418A0000000
ret i1 %1
  =>
%0 = fcmp ole float %x, 0x3FE12878E0000000
ret i1 %0
\end{verbatim}
\end{quote}}

The original code multiplies a floating-point value \texttt{\%x} by a
constant, and then checks if the result is less than another constant.
\minotaur{} found that this code is equivalent to checking if \texttt{\%x}
is less than or equal to a third constant.
It is tricky to reason about floating-point literals, and \minotaur{} is able to
reason and synthesize the correct literals correctly.

\paragraph*{Example 9}

This code comes from 638.imagick in SPEC CPU 2017:

{\small\begin{quote}\begin{verbatim}
%0 = fmul float %x, 0.000000e+00
%1 = fmul float %0, 3.000000e+00
ret float %1
  =>
%0 = fmul float %x, 0.000000e+00
ret i1 %0
\end{verbatim}
\end{quote}}

The original code multiplies a floating-point value \texttt{\%x} by
zero, and then multiplies the result by 3.0. \minotaur{} found that this
code is equivalent to multiplying \texttt{\%x} by zero directly.
Note the original code cannot be optimized to 0.0 directly, because of
the NaN and signed zero propagation rules in floating-point arithmetic.
This example shows that \minotaur{} is able to reason about these corner
cases and synthesize the correct code.

% \paragraph*{Example 10}


%TODO: Add one final example
% place holder for a good example






\section{Related Work}
The message passing interface (MPI)~\cite{dongarra2013mpi} is a
widely-used standardized abstraction for communication primitives in a
multi processor system. Implementations of MPI provide reliable and
portable implementations of collective primitives. Efficient
algorithms for implementing these primitives is a long-studied
research area~\cite{pjevsivac2007performance, chan2007collective,
thakur2005optimization}, including optimized algorithms for specific
architectures like mesh, hypercube, or
fat-tree\cite{scott1991efficient,bokhari1992complete,barnett1993global}
and for clusters of shared-memory
processors~\cite{sistare1999optimization,traff2002improved,sanders2002hierarchical,tipparaju2003fast}.
The class of $k$-synchronous algorithms studied in this paper is
designed to include many of the algorithms proposed in these works and
implemented in popular MPI implementations such as
MPICH~\cite{thakur2005optimization} and
OpenMPI~\cite{gabriel2004open}.

We evaluated OpenMPI, either through builtin CUDA capability or
through Unified Communication X~(UCX)~\cite{ucx}. They lack custom
implementations for architectures such as the \dgxone{}, and result in
subpar performance compared with our NCCL baselines. NCCL~\cite{nccl}
is a library for multi NVIDIA GPU systems and it utilizes the
underlying hardware transport such as NVLink, NVSwitch or Infiniband
for an efficient implementation of collective primitives.
RCCL~\cite{rccl} is a port of NCCL for AMD GPUs and the HIP compiler
suite. While these libraries provide efficient implementations for a
limited set of algorithms, \tool{} is able to synthesize a wide range
of algorithms suitable for different input sizes and generate
collective primitives that are not even a part of standard MPI set.

There are also hybrid algorithms~\cite{barnett1994building,
chan2007collective} that switch between latency- and bandwidth-optimal
algorithm along each dimension of a mesh network. However, to the best
of our knowledge, these prior works do not seek to identify algorithms
that are Pareto-optimal for a given topology. In contrast to these
prior works, the goal of this paper is to automatically synthesize
Pareto-optimal algorithms for a given topology.

There are also hierarchical approaches to implement collective
primitives in distributed systems. Horovod~\cite{alex2018horovod}
implements collective primitives by using NCCL locally in node and MPI
across nodes. Others such as BlueConnect~\cite{blueconnect} and
PLink~\cite{plink} exploit the hierarchical network topology of a
cloud system or a data center to improve the performance of collective
primitives. In this paper, we focus on synthesizing algorithms for a
single node with multiple GPU, while the above approaches are
beneficial on multi node systems.

Motivated by recent resurgence in machine-learning workloads, recent
research has focused on optimizing the communication of distributed
machine learning. Blink~\cite{wang2020blink}, the closest to our work,
automatically synthesizes bandwidth-efficient collective primitives
for a given topology. This work is based on packing spanning trees and
is suitable for one-to-many collective primitives such as broadcast
and reduce, and implements \allreduce as a reduce followed by a
broadcast. Blink is not guaranteed to generate bandwidth-optimal
algorithms as it heuristically selects a few trees based on an
approximate spanning-tree packing algorithm. Moreover, Blink's focus
is not on generating latency-optimal algorithms. In contrast, this
work generates latency- and bandwidth-optimal algorithms for a given
topology. There are also other
works~\cite{zhang2017poseidon,hashemi2019tictac,jayarajan2019priority,peng2019generic}
on optimizing distributed machine learning that do so by overlapping
computation and communication and are orthogonal to this work.

%\todo{Compare to related work on synthesizing compute kernels.
%Orthogonal.}

%\todo{Compareasd to related work on pipelining compute and
%communication. Orthogonal.}

%%% Local Variables: %% mode: latex %% TeX-master: "paper" %% End:

\section{Conclusion}
This chapter introduces \tool: a systematic method to synthesize
algorithms in the Pareto-frontier spanning from the latency-optimal
algorithm to the bandwidth-optimal algorithm for a given collective on
an input topology. We characterize a class of algorithms that captures
a broad set of known algorithms and prove Pareto-optimality of both
known algorithms and synthesized new algorithms. We automatically
generate an implementation of these algorithms that is competitive
with manually hand-tuned communication kernels in use today.


%%
%% The next two lines define the bibliography style to be used, and
%% the bibliography file.
\bibliographystyle{ACM-Reference-Format}
\bibliography{references}


%%
%% If your work has an appendix, this is the place to put it.
\appendix
\section{Artifact Appendix}
\subsection{Abstract}
This artifact contains the source files for \tool. \tool{} has two parts; a synthesizer for synthesizing the optimal communication schedules 
for a given topology and a code generator that lowers the synthesized schedule to CUDA code. 
The synthesizer and the code generator can be executed on any modern x86-64 computers but
the evaluation of lowered code requires a system with CUDA-enabled GPUs and peer-to-peer access. The lowered code
in this paper was evaluated on an NVIDIA \dgxone and a Gigabyte Z52 system. 

This artifact provides instructions on 
how to set up the environment, build and launch the docker image and do a test run of \tool{}. 
It will also give the command lines to reproduce the results in paper, and finally, it will discuss
how to use \tool{} to synthesize schedules for custom topologies and parameters.

\subsection{Artifact check-list (meta-information)}
{\small
\begin{itemize}
\item {\bf Algorithm: } \tool
\item {\bf Program: } \MakeLowercase{\tool.py} is a python script that automatically synthesizes communication schedules and
	lowers them to CUDA source code.
\item {\bf Compilation: } Each generated code comes with a Makefile which requires NVCC and 
	MPICC for compilation.
\item {\bf Run-time environment: } Linux operating system, CUDA run-time, and MPI run-time.
\item {\bf Hardware: } An NVIDIA \dgxone with 8 V100 GPUs connected with NVLinks and a 
	Gigabyte Z52 system with 8 MI50 GPUs connected with PCI and xGMI.
\item {\bf Metrics: } Evaluating the synthesis time and the latency of generated collective 
	communication primitives.
\item {\bf Output: } A schedule for transferring buffers of data required for the desired collective communication
	primitive along with the lowered CUDA code.
\item {\bf Experiments: } Code generation for different versions of 
	\allreduce, \allgather, and \alltoall on different topologies and executing them.
\item {\bf Publicly available?: } The code is available per request.
\end{itemize}

\subsection{Description}
\subsubsection{How delivered}
The source code can be accessed through Github\footnote{\url{https://github.com/parasailteam/nccl/blob/synthesizer/ppopp-ae/sccl-artifact.tar.gz}}.

\subsubsection{Hardware dependencies}
Out experiments were evaluated on an NVIDIA \dgxone and a Gigabyte Z52 system. The \dgxone machine
has 8 V100 NVIDIA GPUs with NVLink connection among them and has dual Intel Xeon E5-2698 v4 processors with a total of 512 GB host memory. 
The Z52 system consists of 8 MI50 AMD GPUs connected via xGMI and PCI and runs with dual AMD EPYC 7002 processors with a total of
1TB host memory.
%The code can be executed on any CUDA capable GPUs with peer-to-peer communication capabilities. 
%This includes systems with NVIDIA GPUs connected via PCI, NVLink, or NVSwitches or AMD GPUs connected with xGMI or PCI.

\subsubsection{Software dependencies}
Our experiments were evaluated on Ubuntu version 20.04, kernel version 4.19
with NVIDIA Docker version 2.5.0, CUDA version 10.2, OpenMPI version 4.0.2, Python version 3.8.5 and Z3 version 4.8.8, and the performance of our generate code 
are compared with NCCL version 2.7.8-1. CUDA, OpenMPI, Python, Z3 and NCCL are automatically installed when building the docker image.

\subsection{Installation}
The installation is done through docker. The \texttt{build\_docker.sh} in
the downloaded tar file includes all of the required steps to get the docker container running.

\subsection{Evaluation and expected result}

\subsubsection{Evaluating the synthesizer}
\tool{} can be queried to synthesize different collective communication primitives. For example,
it can synthesize an \allgather algorithm with 6 chunks in 7 steps and 7 rounds on a \dgxone. 
This will take a few seconds to execute and find the schedule for sending the chunks across
the network. Command line for this example is given in the README.

The output of \tool's shows the synthesized schedule which follows the following pattern
\begin{verbatim}
send 1 from 0 to 3 at step 0
send 2 from 0 to 2 at step 0
send 3 from 0 to 1 at step 0
\end{verbatim}
The output specifies when and what chunk is communicated between a pair of GPUs. GPU $i$'s chunks are identified by $[6i, 6i+5]$ (6 chunks per GPU)
where $i\in[0,7]$ is one of the $8$ GPUs.

An adjacency matrix is followed, displaying the topology of \dgxone where rows (columns) correspond to sources (destinations) and the value 
represents the number of parallel chunks that can be transferred from the source GPU to the destination GPU in a round.

The bandwidth utilities per step are displayed after the topology matrix. This corresponds to the schedule that \tool{} synthesizes for 
each step. A value at row $s$ and column $d$ represents the number of chunks sent from GPU $s$ to GPU $d$ and 
normalized by the number of rounds in that step. This is limited by the entry of the
topology matrix at $(s,d)$. After the bandwidth utilization matrix, an overall link utilization is displayed.

Once \tool{} synthesized the schedule for the communication, a CUDA implementation following the schedule is generated. 

Table~\ref{fig:dgxone:syn} and \ref{fig:amd:syn} can be generated by following the README file.

\subsubsection{Evaluating the generated CUDA code}
This section describes the instructions for reproducing the numbers in Figures~\ref{fig:dgx1-res-allgather}, 
 \ref{fig:dgx1-res-allreduce}, \ref{fig:dgx1-res-alltoall} and \ref{fig:amd-res-allgather}.
The OSU Micro-Benchmarks (OMB) was adopted for the performance evaluation of the generated CUDA code. 
The instructions for executing the generated CUDA code through OMB is given in the README file.

\subsection{Experiment customization}
\tool{} can synthesize collectives with customized topologies, chunks, steps and rounds by expressing
the network topology and setting the command line arguments. The README file includes the instructions.

\end{document}
\endinput
%%
%% End of file `sample-sigplan.tex'.
