
\section{Related Work}
The message passing interface (MPI)~\cite{dongarra2013mpi} is a widely-used standardized abstraction for communication primitives in a multi processor system. Implementations of MPI provide reliable and portable implementations of collective primitives. Efficient algorithms for implementing these primitives is a long-studied research area~\cite{pjevsivac2007performance, chan2007collective, thakur2005optimization}, including optimized algorithms for specific architectures like mesh, hypercube, or fat-tree\cite{scott1991efficient,bokhari1992complete,barnett1993global} and for clusters of shared-memory processors~\cite{sistare1999optimization,traff2002improved,sanders2002hierarchical,tipparaju2003fast}. The class of $k$-synchronous algorithms studied in this paper is designed to include many of the algorithms proposed in these works and implemented in popular MPI implementations such as MPICH~\cite{thakur2005optimization} and OpenMPI~\cite{gabriel2004open}.

We evaluated OpenMPI, either through builtin CUDA capability or through Unified Communication X~(UCX)~\cite{ucx}.
They lack custom implementations for architectures such as the \dgxone{}, and result in subpar performance compared with our NCCL baselines.
NCCL~\cite{nccl} is a library for multi NVIDIA GPU systems and it utilizes the underlying hardware transport such as NVLink, NVSwitch or Infiniband for an efficient implementation of collective primitives. RCCL~\cite{rccl} is a port of NCCL for AMD GPUs and the HIP compiler suite. While these libraries provide efficient implementations for a limited set of algorithms, \tool{} is able to synthesize a wide range of algorithms suitable for different input sizes and generate collective primitives that are not even a part of standard MPI set.

There are also hybrid algorithms~\cite{barnett1994building, chan2007collective} that switch between latency- and bandwidth-optimal algorithm along each dimension of a mesh network. However, to the best of our knowledge, these prior works do not seek to identify algorithms that are Pareto-optimal for a given topology. In contrast to these prior works, the goal of this paper is to automatically synthesize Pareto-optimal algorithms for a given topology.  

There are also hierarchical approaches to implement collective primitives in distributed systems. Horovod~\cite{alex2018horovod} implements collective primitives by using NCCL locally in node and MPI across nodes. Others such as BlueConnect~\cite{blueconnect} and PLink~\cite{plink} exploit the hierarchical network topology of a cloud system or a data center to improve the performance of collective primitives. In this paper, we focus on synthesizing algorithms for a single node with multiple GPU, while the above approaches are beneficial on multi node systems.

Motivated by recent resurgence in machine-learning workloads, recent research has focused on optimizing the communication of distributed machine learning. Blink~\cite{wang2020blink}, the closest to our work, automatically synthesizes bandwidth-efficient collective primitives for a given topology. This work is based on packing spanning trees and is suitable for one-to-many collective primitives such as broadcast and reduce, and implements \allreduce as a reduce followed by a broadcast. Blink is not guaranteed to generate bandwidth-optimal algorithms as it heuristically selects a few trees based on an approximate spanning-tree packing algorithm. Moreover, Blink's focus is not on generating latency-optimal algorithms. In contrast, this work generates latency- and bandwidth-optimal algorithms for a given topology. There are also other works~\cite{zhang2017poseidon,hashemi2019tictac,jayarajan2019priority,peng2019generic} on optimizing distributed machine learning that do so by overlapping computation and communication and are orthogonal to this work. 

%\todo{Compare to related work on synthesizing compute kernels. Orthogonal.}

%\todo{Compareasd to related work on pipelining compute and communication. Orthogonal.} 

%%% Local Variables:
%%% mode: latex
%%% TeX-master: "paper"
%%% End:
