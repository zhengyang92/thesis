%%% -*-LaTeX-*-

\chapter{Background}
\label{chap:background}


\section{Superoptimization}
\label{sec:superoptimization}


Superoptimization is a technique in computer science aimed at finding
the optimal sequence of instructions for a particular task. Unlike
traditional optimization methods that rely on heuristics or general
rules to improve code efficiency, superoptimization systematically
searches the space of possible program transformations to identify the
absolute best sequence of instructions, often for a specific hardware
architecture. This can result in significantly more efficient code,
sometimes surpassing what expert programmers or compilers can achieve.


The concept of superoptimization was first introduced by Alexia
Massalin in a 1987 paper, where she described a system for generating
the shortest possible sequence of machine instructions that performed
the same function as a given piece of assembly code. The
superoptimizer would test all possible combinations of instructions
and compare the output against the original function to ensure
correctness. 





\section{SIMD}
\label{sec:simd}

SIMD stands for Single Instruction, Multiple Data, a type of parallel
computing architecture found within a CPU or GPU. It allows for the
execution of the same operation on multiple data points
simultaneously, making it highly effective for tasks that require the
same processing to be applied to large sets of data. Here's a detailed
look at SIMD and its key aspects:


SIMD exploits data-level parallelism by applying a single instruction
to multiple data points at once. This is different from multiple
instruction, multiple data (MIMD) architectures where different
processors execute different instructions on different data. By
processing multiple data points simultaneously, SIMD can significantly
speed up computations, especially in applications that involve large
arrays of data such as images, audio, and video streams.

SIMD operations are often carried out by vector processors or vector
units within a CPU. These processors have registers that can hold
multiple data elements and execute vectorized instructions on them.

Many CPU architectures include SIMD instruction set extensions that
enhance their ability to perform specific types of parallel
operations. Common examples include: Intel’s MMX and SSE (Streaming
SIMD Extensions): These extensions add new instructions and larger
registers to handle SIMD operations more efficiently. ARM’s NEON: This
is ARM’s equivalent of SIMD for its processor architectures, used
extensively in mobile devices. AVX (Advanced Vector Extensions) by
Intel: This offers enhanced performance and higher bit rates for
processing, suitable for scientific computations and high-performance
tasks.

\section{SIMD in LLVM}
\label{sec:llvm-vectors}

LLVM is a popular open-source compiler infrastructure that supports
various programming languages and architectures. It provides a
high-level intermediate representation (IR) that can be optimized and
compiled to machine code for different platforms. LLVM includes
support for vector operations, which allow programmers to work with
vectors of data efficiently. Here are some key aspects of vectors
support in LLVM:

LLVM provides robust support for vector operations, essential for
optimizing performance in applications that can benefit from SIMD
(Single Instruction, Multiple Data) capabilities. Here’s a detailed
overview of how LLVM handles vector support:

\begin{itemize}

\item Vector types: LLVM defines vector types that represent arrays of
  elements of the same type. For example, a vector type might be
  defined as <4 x i32>, representing a vector of four 32-bit integers.
  These types can be used in LLVM IR to perform operations on vectors
  of data.

\item Vector operations: LLVM provides a set of vector operations that
  work on vector types. These operations include arithmetic operations,
  logical operations, shuffling, and other vector-specific
  instructions. Programmers can use these operations to write
  high-performance code that takes advantage of SIMD capabilities.

\item Target-specific optimizations: LLVM can generate target-specific
  code that leverages the SIMD capabilities of the underlying
  hardware. This allows LLVM to generate efficient code that takes
  full advantage of the vector units available on modern processors.

\item Vectorization: LLVM includes a vectorizer that can automatically
  transform scalar code into vectorized code. The vectorizer analyzes
  loops and other code patterns to identify opportunities for
  vectorization, improving performance by utilizing SIMD
  instructions.

\item Intrinsic functions: LLVM provides intrinsic functions that allow
  programmers to directly access SIMD instructions provided by the
  target architecture. These intrinsics provide a way to write
  low-level code that takes full advantage of the vector units
  available on the hardware.

\item Optimization passes: LLVM includes optimization passes that can
    transform and optimize vector code. These passes can improve the
    performance of vectorized code by applying various optimizations
    such as loop unrolling, instruction scheduling, and register
    allocation.

\end{itemize}



