%%% -*-LaTeX-*-

\chapter{Background}



\section{Superoptimization}


Superoptimization is a technique in computer science aimed at finding
the optimal sequence of instructions for a particular task. Unlike
traditional optimization methods that rely on heuristics or general
rules to improve code efficiency, superoptimization systematically
searches the space of possible program transformations to identify the
absolute best sequence of instructions, often for a specific hardware
architecture. This can result in significantly more efficient code,
sometimes surpassing what expert programmers or compilers can achieve.


The concept of superoptimization was first introduced by Alexia
Massalin in a 1987 paper, where she described a system for generating
the shortest possible sequence of machine instructions that performed
the same function as a given piece of assembly code. The
superoptimizer would test all possible combinations of instructions
and compare the output against the original function to ensure
correctness.



