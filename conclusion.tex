\chapter{Conclusion}
\label{chap:conclusion}

This section concludes the dissertation by summarizing the contributions.
%
The work established in this dissertation is done on two different optimization
domains: peephole generation and collective communication synthesis.
%
The thesis that this dissertation has supported is that program synthesis can
be used to generate optimized and verifiable code for novel architectures.


This dissertation introduced \minotaur{}, a SIMD-oriented
superoptimizer that automatically
 generates peephole optimizations for LLVM IR code.
%
\minotaur{} has been shown to discover optimizations that are missed by
commodity compilers, and has demonstrated speedups on a variety of
benchmarks.

This dissertation also proposed SCCL, a syntheizer that generates
optimal collective communication algorithms a given hardware topology.
%
SCCL synthesizes algorithms along the Pareto-frontier spanning from
latency-optimal to bandwidth-optimal implementations of a collective.
%
The algorithm generated by SCCL are competitive with hand-optimized
collective communication libraries.


I hope that this work will inspire future research in the area of program
synthesis for performance optimization.
