Modern computer architectures are complex, with a wide range of
features that can be leveraged to optimize software performance.
However, efficiently optimizing software for these architectures
remains a challenging task. Traditional optimization techniques,
such as manual optimization by human experts or optimization by
compilers, may not fully exploit the unique features of novel
architectures, leading to suboptimal performance.

Program synthesis is a promising approach to address this challenge.
At its core, program synthesis searches for programs that meet a
specified set of requirements. Recent advances in SMT (Satisfiability
Modulo Theories) solvers and increased computation power have made
program synthesis a viable choice for code generation. Program
synthesis can generate code specifically tailored to the target
architecture, leveraging domain-specific knowledge and advanced search
techniques to create highly optimized code.

The overall goal of this dissertation is to develop program
synthesizers that efficiently optimize software for emerging
architectures. The thesis statement of my dissertation is that program
synthesis can be used to generate highly optimized code for novel
architectures, outperforming traditional optimization techniques. To
achieve this goal, I develop program synthesizers that can
generate optimized code for a variety of hardware platforms. I
evaluate the effectiveness of these synthesizers by comparing the
generated code with manually optimized code and code generated by
traditional compilers. The results show that program synthesis can
produce code that is faster and more efficient than code generated by
traditional optimization techniques.

I present Minotaur, a superoptimizer that uses program synthesis to
optimize LLVM IR (Low Level Virtual Machine Intermediate
Representation) code. Minotaur extracts program slices from LLVM IR
code, and uses an SMT solver to find optimized versions of these
slices. Minotaur is designed to work within the LLVM optimization
pipeline, and can be used to discover new optimizations that are
missed by commodity compilers.

This dissertation introduces SCCL (Synthesized Collective
Communication Library), a program synthesizer that optimizes
collective algorithms for parallel computation. SCCL uses
domain-specific knowledge about collective algorithms to generate
highly optimized code for specific architectures. SCCL is designed to
be a drop-in replacement for existing collective communication
libraries, such as NVIDIA NCCL (NVIDIA Collective Communications
Library) and AMD RCCL (ROCm Collective Communications Library). SCCL
synthesize novel latency and bandwidth optimal algorithms not seen in
the literature on two popular hardware topologies. This dissertation
also shows how SCCL efficiently lowers algorithms to implementations
on two GPGPU hardware architectures and demonstrate competitive
performance with hand optimized collective communication libraries.