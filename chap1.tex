\chapter{Introduction}
\label{chap:intro}

My thesis is as follows:
%
Utilizing program synthesis can be used in performance optimization
for novel hardware platforms, by automatically generating optimized
and verifiable code sequences and algorithms that are tailored to the
specific features and configurations of the target architecture.
%
Specifically, by leveraging program synthesis techniques, we can
generate optimal collective algorithms that are proven to be correct
and optimal for a specific network topology,
%
and we can also automatically generate verifiably correct peephole
optimizations for floating point and vector instructions that are
missing from a commodity compiler.

\section{Motivation}

Optimizing software for performance is a critical task in modern
computing systems.
%
Better performance can lead to better user experiences, more efficient
resource utilization, and lower costs.
%
Developers and researchers have spent a significant amount of efforts over
the years making applications run faster and more efficiently.


However, performance optimization is a challenging, especially in the
post Moore's Law area, the performance gains from improvements in
single core performance are diminishing, and the complexity of
software is increasing.
%
The emergence of novel architectures, such as novel vector processors
and specialized accelerators, has further complicated the task.
%
These architectures offer unique features and capabilities that can
significantly improve the runtime performance, but leveraging these
features effectively requires specialized knowledge and expertise.

One approach to addressing these challenge is that human experts write
low level code from scratch, such as assembly, to take advantage of
the unique features of the architecture.
%
However, this process is time-consuming and
error-prone. Usually, programs optimized in this way are not
portable and may not be optimal for different architectures.
%
Moreover, the complexity of modern software makes it difficult for
human experts to fully understand and optimize the code, as a result,
such approach usually only applies to small compute kernels, rather than
the entire application.
%
Even with the best human experts, the different in the target





Another approach is to use optimizing compilers. However, existing
compilers may not fully exploit the capabilities of novel
architectures, leading to suboptimal performance.
%
1. Compiler optimizations are not complete: they may not be able to
discover all possible optimizations that can be applied to a program.
Compiler often perform local optimizations within a loop or basic
block, but they may not be able to perform global optimizations that
require the analysis of the entire program.


or the limitations of the compiler's
understanding of the target architecture.
New instructions and features in novel architectures may not be
supported by existing compilers.
%
2. Compiler optimizations are not always correct, and newly introduced
optimizations are more likely to be incorrect, leading to incorrect
code generation.
%
3. Compiler optimizations are not always efficient: although compilers
may use different heuristics to determine the best optimization for an
architecture, these heuristics may not be able to fully capture the
characteristics of the architecture.
%
4. Optimization goals are not always aligned: Some
optimizations may interfere with each other or have trade-offs in
terms of code size, execution speed, or memory usage. Compilers may
prioritize certain optimizations over others based on their
cost-benefit analysis.

























Furthermore, even with the best compilers and human experts, the
different in the target architecture may require different
optimization goals. For example, a








Furthermore, existing compilers and optimization tools may not fully
support or exploit the capabilities of these architectures, leading to
suboptimal performance. As a result, there is a growing need for
advanced optimization techniques that can automatically generate
highly optimized code tailored to specific hardware architectures.

A key challenge in optimizing programs for new architectures is the
rapid pace of innovation in hardware design. Novel architectures
introduce new instructions and features that can significantly improve
performance, but existing compilers and human experts may
struggle to fully utilize these features. Optimizing compilers rely on
heuristics to determine the best optimization strategies, which may not
be tailored to the specific features of new architectures. Human
experts can manually optimize code, but this process is time-consuming
and may not fully exploit the capabilities of the hardware.






Some of the challenges in optimizing programs for new architectures
include:

\begin{itemize}
\item \textbf{Architecture-specific features}: New architectures often
\end{itemize}


With the recent advances in SMT solving and program synthesis
techniques, there is an opportunity to develop tools and frameworks
that can automatically generate optimized code for novel
architectures.


\section{Dissertation Overview}



\section{Contributions}

The main contributions of this dissertation are as follows:

\section{Dissertation Outline}

The rest of this dissertation is organized as follows. In Chapter 2, we
provide background information on superoptimization, SIMD architectures,
and program synthesis. In Chapter 3, we present a verification framework
for LLVM vector instructions. In Chapter 4, we introduce Minotaur and
describe its architecture and implementation. Chapter 5 presents SCCL,
a synthesizer for collective communication algorithms. Finally in Chapter
6, we conclude with a summary of our contributions and directions for
future work.





\begin{comment}

The rapid evolution of computer hardware introduces novel
architectures that demand efficient utilization for optimal
performance. These innovative architectures enhance the capabilities
of processors, enabling them to perform more complex tasks and improve
overall efficiency.
%
For example, GPUs achieve high performance by utilizing specialized
hardware features specifically designed to optimize parallel
processing and computational tasks, and the Vector Processing Units
(VPUs) within CPUs are designed to handle vector operations
efficiently, making them particularly well-suited to applications such
as video encoding, compression, and image processing applications.
%
These cutting-edge architectures are gathering significant
interest from software developers and researchers.

Programs running on top of these new architectures are either
optimized manually by human experts, or optimized by optimizing
compilers. However, both approaches struggle to keep pace with
innovations in architecture, as the features in those new
architectures may not be fully supported or understood by existing
compilers and experts;
%
additionally, optimizing compilers and experts usually rely on
heuristics to determine the best optimization strategies.
%
These heuristics might not be tailored to the specific features of
novel architectures, leading to suboptimal performance.
%
Consequently, efficiently optimizing programs for new architectures
remains a challenging endeavor.

Program synthesis is a promising approach to address the challenge.
%
At its core, program synthesis searches for programs that meet a
specified set of requirements.
%
Nowadays, the advances in SMT solvers and increased computation power
have made program synthesis a more viable choice for code generation.
%
There are several reasons why program synthesis can help address the
challenge:

\begin{itemize}
\item \textbf{Adaptability}: Unlike optimizing compilers or human
experts, program synthesis can generate code specifically tailored to
the target architecture.
%minotaur
For example, in superoptimizers, once the semantics of an instruction
is fed into the refinement checker, it can automatically generate code
that takes advantage of the instruction.

\item \textbf{Domain-specific optimizations}: Program synthesis can
take into account the unique characteristics of a specific domain,
such as collective algorithms in parallel computation, to create
highly optimized code.
%sccl
This domain-aware optimization further enhances the performance of
applications on new architectures.

\item \textbf{Continuous improvement}: As an architecture evolves and
new features are introduced, program synthesis techniques can be
updated to incorporate these advancements, ensuring that the generated
code remains optimized over time.
\end{itemize}

This proposal focuses on developing program synthesizers that aim to
efficiently optimize software for emerging architectures. The thesis
statement of my dissertation is

\hfill

\noindent\fbox{%
\parbox{\textwidth}{%
Program synthesis can help developers to better utilize novel
instructions and features on emerging architectures. By leveraging
program synthesis techniques, we can synthesize optimal collective
algorithms that are tailored to a specific topology, and we can also
automatically generate peephole optimizations for vector instructions.
} }

\indent

My dissertation will be organized as follows: the first part will
discuss the work on \emph{SCCL}, which is a collective algorithm
synthesizer, the second part will discuss the work on \emph{Minotaur},
a synthesizing superoptimizer for vector instructions, and the third
part will discuss the pruning techniques specifically for vector
instructions that Minotaur can use to improve its synthesis
performance.

\section{Part 1, Synthesizing Optimal Collective Algorithms}

\fbox{%
\parbox{\textwidth}{%
Note: This work was published in the attached paper
\textit{Synthesizing Optimal Collective Algorithms}. The paper was
presented at PPoPP'21 and got the best paper award. } }

\indent

The convergence of recent trends of training and deploying large
models, and the stagnation of processors' performance due to Moore's
Law, has prompted system designers to incorporate innovative
high-bandwidth interconnect networks within and between nodes in
distributed clusters. For example, the NVIDIA DGX-1 server features
two x86 processors and eight GPUs, connected by NVIDIA's NVLink
network. These network designs are driven not only by the need for
efficient data transfer, but also by hardware factors such as signal
integrity, cooling, and physical layout. These novel topologies
require novel communication kernels to maximize performance.
Currently, these kernels are created and optimized manually.


In this work, I, with my collaborators, proposed a systematic approach
to automatically synthesize high-performance communication kernels. We
developed a program synthesizer, called \emph{SCCL}, that, given a
topology, specified as a graph with bandwidth constraints on nodes and
edges, and a communication primitive, specified as the pre- and post-
condition on data locations, it generates a quantifier-free SMT
formula that captures the set of all feasible algorithms that
implement the primitive on the input topology.
%
Exploring this space to appropriately minimize the number of
communication steps or decrease the granularity of communication at
each step, is a computationally difficult problem.
%
The Z3 solver is invoked to synthesize algorithms that explore this
tradeoff along the Pareto frontier between latency-optimality and
bandwidth-optimality.
%
For every solution from the SMT solver, I implemented the code
generator that lowers the solution into high-performance
implementations.

We synthesized novel latency and bandwidth optimal algorithms not seen
in the literature on two popular hardware topologies. We also show how
SCCL efficiently lowers algorithms to implementations on two systems
(NVIDIA \mbox{DGX-1} and Gigabyte AMD Z52) and demonstrate competitive
performance with hand-optimized collective communication libraries.

\textbf{Proposal}: This work was finished when I was an intern at
Microsoft Research. I am one of two co-first authors, and I
contributed most of the heavy-lifting work. I have confirmed with the
co-authors of the paper that this paper can be used as a section in my
dissertation.

\section{Part 2, Minotaur: A SIMD-Oriented Synthesizing Superoptimizer}

\fbox{%
\parbox{\textwidth}{%
Note: The details of this work can be found in the attached paper
\textit{Minotaur: A SIMD-Oriented Synthesizing Superoptimizer}. The
paper will be submitted to ASPLOS'24. } }
\hfill

Generating high-quality code for vector instruction set extensions
remains a challenge. Extracting parallelism and mapping source-level
constructs onto semantically rich SIMD instructions can be difficult
when dealing with programs written in high-level languages.
Conversely, writing vector code in assembly is time-consuming and
costly, and hinders compatibility with a wide variety of platforms.
%
A widely adopted compromise involves writing high-level code but using
SIMD intrinsics in performance-critical loops. However, this approach
has its own drawbacks, such as an impedance mismatch where mid-level
compiler optimizations lack the necessary semantics for intrinsic
functions and thus cannot optimize them effectively.

I developed Minotaur, a synthesis-based superoptimizer for the LLVM
intermediate representation (LLVM IR).
%
with an emphasis on supporting LLVM's portable vector operations and
Intel-specific intrinsics.
%
The objective is to automatically identify valuable optimizations that
LLVM is currently unable to perform.
%
Due to the significant compilation time required by Minotaur's
search-based approach, the primary target audience is compiler
developers, who can subsequently implement the discovered missing
transformations.


Minotaur builds on the Alive2~\cite{alive2} translation validation
 tool for LLVM IR, to which I added support for 165 vector
 instructions, and LLVM-MCA, which I  use to build a cost model.
%
On top of these foundations, I developed a synthesis engine that
searches for improved code sequences, a cache that stores previously
discovered optimizations, and an LLVM plugin that operates as a
middle-end optimization pass.


Minotaur finds many profitable optimizations that are missing from
LLVM, and I am actively sending patches that implement those new
optimizations to the LLVM community. Minotaur achieves speedups on
various benchmarks and real-world applications, including the integer
parts of SPEC CPU2017, GMP, and libYUV. Minotaur gets an average 4\%,
0.5\% and 1.1\% speedups on GMP, libYUV and SPEC CPU17, respectively.
The maximum speedup on real-world applications is 15\% in GMP.



\textbf{Proposal}: I submitted a paper on this work (as mentioned
above) to PLDI'23, and it got rejected. I am working on improving the
paper.

\section{Part 3, Speeding up Minotaur with Dataflow Analysis Pruning}

This work is a continuation of the work in the previous section. This
work aims to improve the scalability of Minotaur by pruning the search
space using dataflow analysis.

\subsection{Problem}

Superoptimization is computationally hard due to several inherent
 challenges associated with automatically discovering optimal code
 sequences that are semantically equivalent to the input program but
 have improved performance. Some of the primary reasons for the
 complexity of superoptimization are:


\begin{itemize}
\item Exponential search space growth: As the depth of the search
increases, the number of synthesis candidates grows exponentially.
This exponential growth in the search space can make it
computationally infeasible for superoptimizers to explore all
potential optimizations within a reasonable time frame.
\item The complexity of refinement check and constant synthesis
 increases with the size and the complexity of the specification.
\end{itemize}

These limitations can make it difficult for synthesizing
superoptimizers to effectively optimize large and complex codebases.
In reality, I limit the Minotaur's search depth to 3 instructions, and
Minotaur still takes days to finish optimizing real-world
applications. This motivates us to explore ways to prune the search
space.

Mukherjee et al.~\cite{pruning} presents a novel approach to improve
the efficiency and scalability of superoptimization by leveraging
dataflow pruning techniques. The key idea is to compute dataflow facts
of the specification and candidate with the goal of establishing a
conflict between them. If a conflict is found, then the candidate is
pruned from the search space.

I experimented with the dataflow analysis pruning technique in
Minotaur. In the experiment, I selected the known-bits analysis, which
computes the known bits of an LLVM value; if any bit of the
specification is known to be different from the candidate, then the
candidate is pruned. I used the LLVM's implementation of known bits
analysis. However, the known-bits analysis only discarded a tiny
fraction of search space (14 pruned out of 1087).

The reason for the poor pruning performance in Minotaur is that most
of the vector instructions' transfer functions are missing from LLVM
That is, LLVM always returns "unknown" for these vector instructions.
And not only in the known bits analysis, transfer functions are overly
conservative in other heavily-used dataflow analyses such as
demanded-bits analysis. This motivates me to enhance the transfer
functions for LLVM's vector instructions and intrinsics.

\subsection{Proposed Solutions}

Here is a list of my proposed solutions for improving the transfer
functions of vector instructions.
\begin{itemize}
\item Implement transfer functions of known-bits and demanded-bits
analysis for vector data movement instructions. An example here is the
shufflevector instruction, which is used to shuffle the elements of a
vector. The transfer function for shufflevector is similar with its
concrete semantic: Instead of shuffling the elements of the input
vectors, shuffling the data-flow fact.

\item For vertical trivial vector instructions, I will use the
transfer function of the corresponding scalar instruction and apply it
to each element of the vector.

\item For non-trivial vertical vector instructions, such as
x86.avx2.pavg, which gets the average of two packed integers
vertically.
%
I will compose each basic operations' transfer function to get the
transfer function of the whole instruction.



\end{itemize}.


\subsection{Proposed Evaluation}

I will evaluate the proposed solution by running Minotaur on
real-world applications and macro benchmarks, showing how many
candidates are pruned for every combination of abstract domains
involved, and how many new optimizations are found by enabling
data-flow based pruning.


\section{Proposed Timeline}

\begin{center}
\begin{tabular}{| l | l|}
     \hline
     Proposed Work & Time  \\ %[0.5ex]
     \hline\hline
     Resubmit Minotaur paper to ASPLOS'24 & April, 2023  \\
     \hline
     Implement new transfer functions & May - June, 2023 \\
     \hline
     Generate transfer function by composition & Jun - July, 2023 \\
     \hline
     Write pruning paper & July - August, 2023 \\
     \hline
     Write dissertation & Sept, 2023 onwards \\
     \hline
     Thesis defense & Dec, 2023 onwards  \\ %[1ex]
     \hline
    \end{tabular}
 \end{center}

\section{Expected Research Contribution}

\begin{itemize}
\item A systematic approach to synthesizing collective communication
algorithms that are explicitly tailored for a particular hardware
topology.
\item A SIMD-oriented superoptimizer that can automatically discover
and implement missing optimizations for SIMD instructions in LLVM.
\item A pruning method that can significantly improve the scalability
of a SIMD-oriented superoptimizer.
\end{itemize}

\section{Literature Review}

A \emph{superoptimizer} is a type of program optimizer that utilizes
search techniques to generate more efficient code, setting it apart
from traditional compilers that apply a predetermined (although
potentially extensive) sequence of transformations.
%
The term \emph{superoptimizer} originated from Massalin's
work~\cite{massalin}, which described an exhaustive generation of
machine instruction sequences, employing various strategies to trim
the search space and using testing to eliminate unsuitable candidates.
Preceding contemporary solver-based approaches, Davidson and
Fraser~\cite{peep84} developed peephole optimizations derived from
machine description files.
 %
Modern superoptimizers, in contrast, leverage solvers for automated
reasoning about program semantics, harnessing their capabilities to
optimize code more effectively than earlier methods.

Minotaur is heavily influenced by Bansal and Aiken~\cite{Bansal06},
which presented a superoptimizer that operated on x86 assembly code
and effectively utilized vector instructions.
%
Starting with unoptimized assembly code generated by GCC, their
superoptimizer was able to produce code that competed with higher
optimization levels.
%
The overall structure of this superoptimizer, which involved
extracting program fragments, canonicalizing them, checking them
against a cache, and optimizing them in case of a cache miss, is
similar to Minotaur. However, there are significant differences in the
details, especially in Minotaur's slice extractor, which enables the
synthesis specification to more closely approximate the original
code's effect.
%
STOKE~\cite{stoke}, another assembly superoptimizer, is not as closely
related to my research. STOKE employs random perturbations on entire
assembly-language functions to optimize the code.

Several recent projects have shifted their focus from optimizing
individual programs to generating program rewrite rules.
%
OptGen~\cite{optgen} discovers scalar peephole optimizations that
conform to a specific syntactic form.
%
Even with small rewrite sizes, it identified numerous optimizations
missing from the GCC and LLVM.
%
VeGen~\cite{vegen} generates SLP vectorization rules—an SLP vectorizer
combines a set of scalar operations into vector instructions. VeGen
parses the Intel Intrinsics Guide to construct pattern matchers for
x86 vector instructions. It applies these pattern matchers to the
input scalar program and substitutes scalar expressions with vector
instructions when a profitable match is found. VeGen employs syntactic
pattern matching instead of solver-based equivalence/refinement
checking.
%
Diospyros~\cite{diospyros} is another vector rewrite rule generator.
It adopts an equality saturation~\cite{equalitysat} and uses a
translation validator to discard unsuitable candidates.
%
As a tool based on equality saturation, Diospyros constructs its
search space using existing rewrite rules.

\emph{Program synthesis} -- generating implementations that adhere to
a given specification, is the essential technique in Minotaur and
SCCL.
%
Rake~\cite{rake} performs instruction selection for vectorized Halide
expressions using a two-stage synthesis algorithm.
%
Cowan et al.~\cite{ml_syn} synthesizes quantized machine learning
kernels, introducing two sketches: a compute sketch for matrix
multiplication and a reduction sketch to gather computation results in
the correct registers.
%
It relies on Rosette~\cite{rosette} to generate an efficient NEON
 implementation that satisfies the specifications for both sketches.
 %
SwizzleInventor~\cite{swizzleinventor} synthesizes data movement
instructions for a GPU compute kernel. SwizzleInventor requires
user-defined sketches to describe the non-swizzle part of the program.

\section{Conclusion}

This research proposal outlines an approach to leveraging program
synthesis for efficient optimization of novel instructions on emerging
architectures. The project's success will contribute to the
development of highly optimized, adaptive, and intelligent code
generation techniques, enabling improved performance on cutting-edge
hardware systems.

\end{comment}